\documentclass[12pt]{report}
\usepackage{textcomp}
\usepackage{amsfonts}
\usepackage{amssymb}
\usepackage{makeidx}
\usepackage{amsthm}

% Given that this is a document, it should be relatively easy to determine the
% nature of this document, it's author, &tc; The class was in Fall '04. I
% begain writing this in early Jan '05. I hope to finish it before the end of
% Jan, since I'll have to concentrate on other matters at that point.

\title{
Notes on Topology
}
\author{
    Keith Jones\\
    {\tt kjones@math.binghamton.edu}\\
}

% paragraph headers
\newcommand{\example}{  \noindent{\sc Example }\hspace{5pt} }
\newcommand{\exercise}{ \noindent{\sc Exercise }\hspace{5pt} } 
\newcommand{\define}{  \noindent{\sc Definition }\hspace{5pt} }
\newcommand{\claim}{ \noindent{\sc Claim }\hspace{5pt} }
\newcommand{\corollary} {\noindent{\sc Corollary}\hspace{5pt} }

% shorthand for shorthand symbols
\newcommand{\fall}{\forall\,}
\newcommand{\exst}{\exists\,}
\newcommand{\st}{\textrm{ s.t. }}
\newcommand{\nin}{\not \in}
\newcommand{\contra}{$(\Rightarrow\Leftarrow)$}

% shorthand for set theory stuff
\newcommand{\unionover}[2]{\bigcup_{#1 \in #2 }}
\newcommand{\intover}[2]{\bigcap_{#1 \in #2}}
\newcommand{\productover}[2]{\prod_{#1 \in #2}}
\newcommand{\relate}{\ \mathcal{R}\ } % find a better symbol?

% oft-used sets
\newcommand{\reals}{\mathbb{R}}
\newcommand{\naturals}{\mathbb{N}}
\newcommand{\rationals}{\mathbb{Q}}
\newcommand{\integers}{\mathbb{Z}}
\newcommand{\irrationals}{\mathbb{I}}


% shorthand for topology stuff
\newcommand{\T}[1]{$\textrm{T}_#1$}
\newcommand{\interior}[1]{#1^{\circ}}
\newcommand{\eball}{B_\varepsilon}
\newcommand{\dball}{B_\delta}
\newcommand{\ball}[1]{B_{#1}}
\newcommand{\triineq}{$\bigtriangleup \leq$}
\newcommand{\comp}[1]{\mathcal{C}_#1}
\newcommand{\inv}[1]{#1^{-1}}

\newtheorem{theorem}{Theorem}[section]
\newtheorem{lemma}[theorem]{Lemma}
\newtheorem{axiom}{Axiom}


\makeindex

\begin{document}

\maketitle

\tableofcontents


\chapter{Introduction}

\section{About This Document}
These notes were recorded during the Fall of 2004 at SUNY Binghamton in a
class on point-set topology with Professor Matthew Brin. Part of the grade
requirement for the class was to rewrite and submit the notes taken during
class. The purpose for this was twofold: first, by rewriting our notes, we
were forced to reexamine them and make sense of them; second, the rewritten
notes would serve as a reference later on. \\

\noindent
\textbf{Very important notice.}\\
This document is therefore a copy
of a revision of a transcription; some percentage of the words are taken
directly from the mouth of the professor (these may or may not be denoted as
unreferenced quotations), and some from the mind of the
student. Not all of it has been reviewed, so there is a definite possibility
of error. In fact, the likelihood of finding an error upon any serious perusal
probably hovers around 100\%. Furthermore, while a minimal
 amount of the math may be ``original''
(i.e. not transcribed from a lecture or shamelessly stolen -- I don't mean to
imply there is anything novel in these notes), any errors should be considered
misinterpretation, misunderstanding, or some such error on my part. I should
also note that as of yet, I have been way to lazy to add any drawings or
figures. That may never happen in fact.


\section{Introduction to the Subject Matter}
A typical course in mathematics will introduce the students to {\em definitions},
{\em theorems}, aend {\em examples}. Often there is a ``key'' definition, around which the
course revolves. For example, in this course it would be:\\
\\
\define A {\em topological space} is a pair $(X, T)$ where $X$ is a set and
$T$ is a collection of subsets of $X$ satisfying: 
\begin{enumerate}
\item $\emptyset \in T$.
\item $X \in T$.
\item $A, B \in T \Rightarrow A\cap B \in T$.
\item $A_{\alpha}, \alpha \in I, \in T \Rightarrow \bigcup A_{\alpha} \in T$.
\end{enumerate}


The reason we begin with this definition, even though it may not be the most
appropriate to begin with, is that it illustrates an important point: it
simply does not reveal much about what topology actually is. It is simply the
typical, possibly minimal, definition which allows mathematicians to begin
proving theorems in topology. It is not even the most useful definition, since
the majority of theorems require that a Hausdorff topology be used.

Also, from the terminology in the definition, it would appear that topology
relies heavily on set theory. Indeed; topology is very low level; it sits just
above set theory.  These notes begin with a brief introduction to (or review
of) logic and set theory, since they are so essential to topology.

\section{Some Symbols}

Symbols denoting common sets of numbers:
\begin{itemize}
\item[$\naturals$] denotes the natural numbers
\item[$\integers$] denotes the integers
\item[$\rationals$] denotes the rational numbers
\item[$\irrationals$] denotes the irrational numbers
\item[$\reals$] denotes the real numbers
\end{itemize}


\noindent
Logic symbols:
\begin{itemize}
\item[$\fall$] -- ``for all''
\item[$\exst$] -- ``there exists''
\item[$\in$] -- ``in'' or ``is in''
\end{itemize}

\section{On Writing Proofs}

Herein I shall attempt to place accrued wisdom regarding the challenges of
creating mathematical proofs. No guarantees.\\

A key element in constructing a good proof is to be as lazy as possible. By
this, I mean that you should postpone thought for as long as possible.
Generally, this means $(a)$ always go back to definitions, since they will
probably be very important in the proof, and $(b)$ develop a set of habitual
practices for common techniques in proofs. For example, when when trying to
prove $A \Rightarrow B$, you assume $A$ and look at $B$, trying to work
backwards toward $A$.

I should note one caveat of $(a)$ above: often you can use a previously proven
theorem or lemma to help you prove something quickly, rather than
going all the way back to a definition. The key is to know what tools are
available to you and use them.

Another good example of a ``common habit'' is, when trying to prove that two
sets are equal (which will come up very often) to immediately think of proving
mutual containment. 

The idea here is that for many simpler proofs in set theory, it would behoove
you to be rather mechanistic: understand and be able to recall the various
rules of logic and definitions of the subject, and allow the hypotheses and 
consequences to develop the proof, if at all possible... maybe what I'm saying
is B.S... dunno. :p Will ponder this more and hopefully add some stuff.

\subsection{Common Mistakes}

Consider the following: \\

\claim If $x$ is a limit point of $A \cup B$, then $x$ is a limit point of $A$
or $x$ is a limit point of $B$.\\

Here is an attempt at a proof: 
 $x \textrm{ is a limit point of } A \cup B   \Rightarrow$ \\
 $\fall U \textrm{ open about } x,\ U \cap ((A \cup B) - x) \neq \emptyset
\Rightarrow$ \\
 $\fall U \textrm{ open about } x,\ \exst z \st z \in U$ and $z \in (A \cup B)
 - \{x\} \Rightarrow$ \\
 $\fall U \textrm{ open about } x,\ \exst z,\ z \neq x$, and $(z \in U \cap A
 \lor z \in U \cap B) \Rightarrow$ \\
 $\fall U$ open about $x$, $\exst z,\ z \in U \cap (A - \{x\}) \lor Z \in U
 \cap (B - \{x\})$ \\
 $\fall U$ open about $x$, $[(U \cap (A - \{x\}) \neq \emptyset \lor U \cap (B
-\{x\}) \neq \emptyset]$ \\

The next logical step one might wish to take is...\\
$\therefore$ $x$ is either a limit point of $A$ or a limit point of $B$. \\

This is wrong! Given predicates $S$ and $T$,\\
% FIX ``not iff'' doesn't work, sadly\\
\begin{displaymath}
[\fall x,\ S(x)] \lor [\fall x, T(x)] \not \iff \fall x,\ [S(x) \lor T(x)]
\end{displaymath}
For example, let $x \in \reals - \{0\}$ and $S(x) := x > 0$ and $T(x) := x <
0$. Clearly, the former expression is false, while the latter expression is
true. Make sure to avoid this kind of mistake.\\

\exercise Give a correct proof of the above claim.\\


Another illustration of a common mistake can be found in the handwritten notes
of 11/3.




\chapter{Logic}
A deep familiarity with logic is an essential requirement for anyone
interested in dealing with proofs; but you cannot teach a person how to
construct proofs with just a course in logic. It requires experience.

Logic begins with {\em statements}. All statements are considered to be either
{\em true} or {\em false}. (There is a lie by omission here, since we can
create statements which our system of logic cannot handle; i.e. "this
statement is false". We ignore these because it suits us to do so.)  This kind
of logic is called Boolean logic, named after George Boole who developed the
system.

In logic, there are two types of statements: {\em
propositions}\index{proposition} and 
{\em predicates}\index{predicate}.
Propositions are demonstrably true or false ``on their own
merits'', while predicates require more information. \\


\example\\
Some propositions: $0 > 2$, $0 < 2$\\
Some predicates: $x > y$, $x + 2 = 7$\\


With predicates, we accept that the true/false value fluctuates (i.e. there is
a variable). A proposition may be assigned a letter, generally uppercase, 
i.e. $A,B,C,...$, while predicates look like functions: $A(x),B(y),$ etc.\\
\\
Note that since we are interested in the logic itself, we may choose arbitrary
letters without relating them in any way to the real world. That is, we are
really talking about nothing. \\
\\
Quantifiers:\\
\indent 
$\forall$ (read "for all") and $\exists$ (read ``there exists'') are shorthand
notation used in symbolic logic. Generally, $\forall\, x A(x)$ is true if every
valid value of $x$ makes $A(x)$ true, and $\exists\, x A(x)$ is true if there is
at least one value value for $x$ which makes $A(x)$ true.  \\
\indent 
By using quantifiers, we create a proposition from a predicate. I.e. $A(x)$, a
predicate, becomes the proposition $\forall\, x A(x)$. This is because, given
some $A$, we can determine (with some effort) the truth value. For xample,
assuming the set of valid values for $x$ is the integers, $\forall\, x x\ge 2$
is clearly false. \\
\indent 
It is important to note, as this demonstrates, that while
a statement may have variables in it, it is not necessarilty a predicate.
Consider the statement $x>y$. This is a predicate. Now, $\exists\, x, x>y$ is
still a predicate, but we say that 
the $x$ is {\em bound}. Since we need information about the $y$ 
before we can determine the truth value of the statement, we say the $y$ is
a {\em free variable}. Finally, if we bind $y$ by stating
$\forall\, y, \exists\, x, (x>y)$, we have a proposition, since this statement can
be shown to be true or false. Notice this statement is true, but if we switch
the quantifiers to get $\exists\, x \forall\, y (x>y)$, we have a false statement;
so the order of the quantifiers is very important.\\
\\

\chapter{Set Theory}

\section{Introduction}

Set theory is an axiomatic system. Axiomatic Systems make use of the
following...
\begin{enumerate}
\item {\em variables} of one or more type. In set theory, there is only one 
type.
\item {\em relation symbols}, which may take infinitely many arguments
\item {\em function symbols}, which may take infinitely many arguments
\item {\em axioms}
\item {\em rules of inference}
\end{enumerate}

The difference between relation and function symbols is that the
result of a relation symbol is a true/false value, while the result
of a function symbol is another variable.

What we will refer to as ``set theory'' is G\"odel, von Neumann, and
Berney's extension of Zermelo-Frankel set theory. It may happen that some
people will say they are using ZF set theory, when they are really using
G-vN-B set thoery.

In this version of set theory, we have:
\begin{enumerate}
\item A single type of variable called a \index{class} {\em class}
\item One undefined relation symbol $\in$, which takes two arguments; written
$x \in y$
\item No undefined function symbols.
\end{enumerate}

\define $x$ is a \index{set} {\em set} if and only if $\exst y \st x \in y$.\\

The gist is that if $x$ is a set, then there must be some class 
(which may or may not be a set) which contains $x$; and if there is some class 
which contains $x$, then $x$ must be a set.

Everything in set theory is a class, but not everything is a set. So all sets
are classes, but not vice-versa.  It is important to note this distinction. 
Without it, we would arrive at Russell's paradox 
(named for Bertrand Russell), as will be discussed shortly.

\section{The Axioms}

\define $x \subseteq y$ is defined as $\fall z,\ (z \in x) \Rightarrow (z \in
y)$\\

\define $x = y$ is defined as $x \subseteq y$ and $y \subseteq x$\\

Consider the following two statements: 
\begin{enumerate}
\item $x \in y \land x = z \Rightarrow z \in y$
\item $x \in y \land y = z \Rightarrow x \in z$
\end{enumerate}

Notice that while the second statement follows directly from the above
definitions, the first does not. This leads to our first axiom.

\index{axiom!of individuality}
\begin{axiom}[of Individuality]
$(x \in y) \land (x = z) \Rightarrow (z \in y)$
\end{axiom}

\index{axiom!of class formation}
\begin{axiom}[of Class Formation]
For each predicate $P(x)$ in which only set variables are quantified and $y$
is never mentioned, there exists a class $y$ such that $\fall x,\ [(x \in y)
\iff x$ is a set and $P(x)]$.
\end{axiom}

This axiom more than any other captures the reason for the existence of set
theory: grouping things together.

With the axiom of formation, we can discuss Russell's Paradox: Pretend there
is no distinction between classes and sets, and consider the
predicate $P(x) = x \not \in x$. By formation, $\exst y \st x \in y \iff
P(x)$. Is $P(y)$ true or false? If $P(y)$, we have  
$y \not in y \Rightarrow y \in y$,
but if not $P(y)$, then $y \in y \Rightarrow y \not \in y$. Both values of y
lead to a contradiction. This is why we need not just sets, but classes
also.\\

The class $\emptyset$ (the empty {\em class}) exists by formation, using the
predicate $P(x) := x \neq x$. Since $x = x$ ($\iff \fall z,\ (z \in x
\Rightarrow z \in x)$, a tautology), we know that $\fall x,\ x \not \in
\emptyset$.

\index{axiom!of the empty set}
\index{empty set}
\index{$\emptyset$}
\begin{axiom}[of the Empty Set]
$\emptyset$ is a set.
\end{axiom}

If $x$ and $y$ are sets and $x \neq y$, then, by formation, there is 
a class $\{x,y\}$ satisfying $z \in \{x,y\} \iff z = x \lor z = y$.

\index{axiom!of pairs}
\begin{axiom}[of Pairs]
$\{x,y\}$ is a set.
\end{axiom}

\index{family!of sets}
\define 
Let $A$ be a function on one variable, defined on some set $I$.
If, for every $\alpha \in I$,
$A(\alpha)$, usually written $A_\alpha$, is a set, then $\{A_\alpha \mid
\alpha \in I \}$ is referred to as a {\em family of sets indexed over $I$}. The
set $I$ is called the \index{indexing set} {\em indexing set}.\\

\define $\bigcup\{A_\alpha \mid \alpha \in I\}$ is defined by 
$\{ z \mid \exst \alpha \in I \st z \in A_\alpha\}$. This is generally written
as $\unionover{\alpha}{I}A_\alpha$.

\index{axiom!of union}
\begin{axiom}[of Union]
If $I$ is a set and for all $\alpha \in I,\ A_\alpha$ is a set, then $\unionover{\alpha}{I}A_\alpha$ is a set.
\end{axiom}

\index{axiom!of replacement}
\begin{axiom}[of Replacement]
If $f$ is a function defined on a set $A$, then the class $f(A)$ built by
formation using $P(X) := \exst a \in A \st x = f(a)$ is a set. The set $f(A)$
is called the \index{image} {\em image of A under f}. 
\end{axiom}

In other words, ``the functional image of a set is a set''.\\

\define If $Q$ is a class, then consider the predicate $S(x) := x \subseteq
Q$. By formation, there is a class $P(Q)$, the power {\em class} of Q such
that $x \in P(Q)$ if and only if $x \subseteq Q$.

\index{axiom!of sifting}
\begin{axiom}[of Sifting]
If $A$ is a set and $C$ is a class, then $A \cap C$ is a set.
\end{axiom}

By formation, $A \cap C$ is a class.


\index{power set}
\index{axiom! of power set}
\begin{axiom}[of Power Set]
If $Q$ is a set, the $P(Q)$, the {\em power set} of $Q$, is a set.
\end{axiom}

\define If $A_\alpha$, a family of sets indexed over some indexing set $I$ 
(REMARK: here I say that I may be a class or a set.... dunno 
whether this or the previous is true, leaning toward this),
then $\intover{\alpha}{I}A_\alpha$ is defined as $\{z \mid \fall \alpha \in
I,\ z \in A_\alpha\}$. Here we assume $I \neq \emptyset$. By sifting, such an
intersection is a set.

\index{axiom!of foundation}
\begin{axiom}[of Foundation]
If $A$ is a non-empty set, then $\exst x \in A \st x \cap A = \emptyset$.
\end{axiom}

That is, in any nonempty set, there must be some element which
is \index{disjoint} {\em disjoint} to that set; i.e. they have no element in
common.\\

\claim This rules out any infinitely descending chains of containment: 
$ \cdots \in x_3 \in x_2 \in x_1$.

\begin{proof}
Suppose, we have $x_1$, $x_2$, $\cdots$, and each $x_i$ is a set. Then their
union is a set $B = \{x_1, x_2, x_3, \cdots\}$, Now suppose each $x_i \in B$
contains $x_{i+1}$, as above. This violates the axiom of foundation since for
no $i$ does $x_i \cap B = \emptyset$. 
\end{proof}

As a consequence,  it is impossible to have $x \in x$
(self-containment) or $x_1 \in ... \in x_1$ (cyclic containment).\\

With the axioms of formation and the axiom of foundation, we can discuss a
little more about the difference between classes and sets. 

Consider the predicate $S(x) := (x = x)$. By formation, there is a class, call
it $C$, such that $x \in C \iff x$ is a set and $(x = x)$. Since $x = x$ is a
tautology, we see that $C$ is the class of all sets.

The question arises; is $C$ a set? If it is, then $C = C$ certainly, so then
$C \in C$. But this cannot be, of course, since then $C$ would contain itself,
which the axiom of foundation prevents. So $C$ is not a set.\\

\define \index{successor} Consider a set $x$. By the above, $x \not \in x$, so
$x \subseteq x \cup \{x\}$ but $x \neq x \cup \{x\}$. We denote $x \cup \{x\}$
by $s(x)$, and call it the {\em successor} of $x$.\\

We can now construct a na\"ive model for the natural numbers:
Let $0 = \emptyset$, $1 = s(0) = \emptyset \cup \{\emptyset\} = \{\emptyset\}$,
$2 = s(1) = \{\emptyset\} \cup \{\{\emptyset\}\} = \{\emptyset,
\{\emptyset\}\}$, $\cdots$

\index{axiom!of infinity}
\begin{axiom}[of Infinity]
There is a set $A$ with the following properties:
\begin{enumerate}
\item $\emptyset \in A$
\item $\fall a,\  a \in A \Rightarrow s(a) \in A$
\end{enumerate}
\end{axiom}

Let $\mathcal{I}$ be a class defined formation using by the above properties;
that is, $A \in I \iff (\emptyset \in A) \land (\fall a,\ a \in A \Rightarrow
s(a) \in A)$, and let $\mathcal{N} = \intover{A}{\mathcal{I}}A$. Then
$\mathcal{N} \in \mathcal{I}$, since $\emptyset \in \mathcal{N}$ and if $a \in
\mathcal{N}$ then $a \in A$ for all $A \in \mathcal{I}$, and so $s(a) \in A$
for all $A \in I$, which means $s(a) \in \mathcal{N}$.

Since for all $A \in I$, $\mathcal{N} \subseteq A$, we have the logical basis
for \index{induction} {\em induction}.\\

\define {\sc (Induction)}. Let $U(n)$ be a predicate valid
for all $n \in \mathcal{N}$.  Let $T(U(n))$ be $\mathcal{N} \cap \{n|U(n)\}$,
by axiom of formation.
$T(U)$ is the set of all elements of $\mathcal{N}$ for which $U$ is true. If
$\emptyset \in T(U)$, and $\fall a,\ a \in T(U) \Rightarrow s(a) \in T(U)$,
then $T(U) \in I$ and therefore $\mathcal{N} \subseteq T(U)$. But $T(U)
\subseteq \mathcal{N}$; hence $T(U) = \mathcal{N}$.

We will revisit this more thoroughly soon.\\ 

\exercise If $x$ is a set, then there exists $\{x\}$ satisfying $z \in \{x\}
\iff z = x$.\\

\index{ordered pair!orig. definition}
\define If $x$ and $y$ are sets, then the {\em ordered pair} $(x,y)$ is 
defined as $\{\{x\},\{x,\}\}$.\\

Notice that by this definition, 
\begin{displaymath}
(0,0) = \{\{0\},\{0,0\}\} = \{\{0\},\{0\}\} = \{\{0\}\} = \{1\}.
\end{displaymath}
Strange, huh?

\section{Functions \& Relations}

\subsection{Functions}
We have yet to adequately describe what we mean when we use the word
function.\\

\define A \index{function!orig. definition} {\em function} is a symbol $f$, such that the
following conditions holds:
\begin{enumerate}
\item The expression $f(x_1, x_2, \cdots, x_n)$ represents a variable.
We call $n$ the \index{arity} {\em arity} of $f$, and $f$ always works
on $n$ variables. 
\item $f$ must be \index{well defined} {\em well defined}. I.e. $x_1 =
x_1^{\prime} \land \cdots \land x_n = x_n^{\prime} \Rightarrow f(x_1, ...,
x_n) = f(x_1^\prime, ... x_n^\prime)$.\\
\end{enumerate}

We will generally look at functions of a single variable, since it suits us
to, but we may at some points look at other kinds of functions.\\

Given sets $A$ and $B$, if $f$ is defined on $A$ and
$\fall a \in A,\ f(a) \in B$, then we write $f:A\rightarrow B$.\\

\define \index{graph} Given some function $f:A\rightarrow B$ for some sets $A$
and $B$, consider the predicate 
$S(x) := \exst a \in A \land x = (a, f(a))$. With formation, we can define a
subset of $A \times B$, which is what we generally think of as the {\em graph}
of a f. We know that the graph is a set, not just a class, by sifting, since
it is a subset of of $A \times B$.\\

In ``na\"ive'' set theory, we don't distinguish between a function and its
graph. The reason we don't bother to is that we could prove that a function is
known if and only if it's graph is known; so they are kind of equivalent. But
we won't do that proof here. With that in mind, an alternative definition of
function...\\

\define A \index{function!alt. definition} {\em function} from a set $A$ to a set $B$ is a
subset of $A \times B$ satisfying:
\begin{enumerate}
\item $\fall a \in A,\ \exst b \in B \st (a,b) \in f$ and
\item $\fall a \in A$, there exists no more than one $b \in B \st (a,b) \in
f$. I.e. $(a,b) \in f \land (a, b^\prime) \in f \Rightarrow (b = b^\prime)$.
\end{enumerate}

The above two conditions constitute an ``existence/uniqueness'' pair, which is
very common through mathematics.

We generally write $f(a) = b$ to mean $(a,b) \in f$. With this definition of
functions, we can talk about such things as intersections of functions,
containment, etc.\\

As discussed earlier, for a function $f:X\rightarrow Y$, given a subset 
$A$ of $X$, we can consider $f(A) \subseteq Y$, where 
$f(A) = \{y \in Y \mid \exst x \in A \st f(x) = y\}$. Recall, $f(A)$
is the image of $f$ under $A$.

We can also look at $f^{-1}$ in this way. Given $B \subseteq Y$, $f^{-1}(B)
\subseteq X$, where by $f^{-1}(B)$ we mean $\{x \in X \mid f(x) \in B\}$. We
call $f^{-1}(B)$ the \index{inverse image} {\em inverse image of B under f}.\\

\subsection{Relations}

\define A \index{relation} {\em relation} from a set $A$ to a set $B$ is a
subset of $A \times B$. \\

All functions are relations, but obviously the converse is not true. For some
relation $r$ with $(a,b) \in r$, we say ``$a$ is related to $b$ by $r$. There
are many common relations, such as ``<'', from $\reals$ to $\reals$. For a
binary relation $r$ (one operating on two variables), we generally write $a\
r\ b$ rather than $(a,b) \in r$. \\

There are many common relations, such as >, <, $\leq$, $\geq$, =, and |.
($|:\integers \rightarrow \integers$, the ``divides'' relation).\\

Our original definition of order pairs is not associative. I.e. $(a,(b,c))
\neq ((a,b),c)$.  Since this can lead to messiness that we don't wan to deal
with, we have an alternative definition.

\index{ordered pair!alt. definition}
\define An {\em ordered pair} is a function whose domain is {0,1}. 

I.e. For $(a,b)$, $f(0) = a$ and $f(1) = b$.\\

We prefer this definition over the previous because it extends well to
n-tuples: 

\index{ordered n-tuple}
\define An {\em ordered n-tuple} is a function whose domain is \{0, 1, ...,
$n-1$\}. \\

So an element of $A_0 \times A_1 \times \cdots \times A_{n-1}$ is a function f
defined on \{0, 1, ..., $n-1$\} so that $f_i \in A_i$ for $0 \leq i \leq
n-1$.\\

\define For $A_\alpha$ a family of sets indexed over indexing set $I$, 
\begin{displaymath}
\productover{\alpha}{I}A_\alpha = \{\textrm{ functions }
f:I\rightarrow \unionover{\alpha}{I}A_\alpha \mid \alpha \in I, f(\alpha)
\in A_\alpha\}.
\end{displaymath}
This is a set since $I$ is a set and all $A_\alpha$ are sets,
so their union is a set. Now, it makes it easier to understand this if we
always try to remember here that these functions are just ways of representing
points which are mathematically convenient. I tend to get myself confused with
the function terminology and start thinking all kinds of wrong things.\\

Now, if each $A_\alpha \neq \emptyset$, is $\productover{\alpha}{I}A_\alpha =
\emptyset$ or not? With the information we have so far, we cannot answer this
question in general. This leads us to our next topic.

\subsection{The Axiom of Choice}
\index{axiom!of choice}
We will describe the axiom of choice in two different forms:

\define {\sc axiom of choice - set form}\\ If $\{A_\alpha\}$, $\alpha \in I$, is
a family of sets indexed over $I$, such that each $A_\alpha \neq \emptyset$,
and they are \index{disjoint!pairwise} {\em pairwise disjoint} (i.e. $\alpha \in I,\ \beta \in I,\ \alpha \neq \beta \Rightarrow A_\alpha \cap A_\beta = \emptyset$), then there is some set $C$ such that $\fall \alpha \in I,\ C \cap A_\alpha$ has exactly one element.\\

\define {\sc axiom of choice - function form}\\ If $\{A_\alpha\}$, 
$\alpha \in I$, is a family of sets indexed over $I$ such that each $A_\alpha
\neq \emptyset$, then there exists a function $f: I \rightarrow
\unionover{\alpha}{I}A_\alpha$, so that for all $\alpha \in I$, $f(\alpha) \in
A_\alpha$.\\

The function form of the axiom of choice tells us the answer to the above
question. If each $A_\alpha$ is nonempty, then there is some function such
that $\productover{\alpha}{I}A_\alpha$ is nonempty.\\

\subsection{More on Functions \& Relations}

\define If $A$ and $B$ are sets, then $A-B = \{x \in A \mid x \nin B \}$. This is often written as $A \setminus B$.

If $B \subseteq A$, then $A - B$ is called the \index{complement} {\em
complement of B in A}. Often it will just be discussed as the ``complement of
$B$'', but in such a case there must be some agreed upon $A$ such that $B
\subseteq A$. Otherwise, the term ``complement'' has no meaning.\\

\define A function $f:A\rightarrow B$ is \index{onto} {\em onto} if 
$\fall b \in B,\ \exst A \in
A \st f(a) = b$.  That is, every element in $B$ is the result of $f$ operating
on some element of $A$.\\

\define A function $f:A\rightarrow B$ is \index{one to one} {\em one to one}
(or 1 -- 1) if $\fall a,a^\prime \in A,\ f(a) = f(a^\prime) \Rightarrow a =
a^\prime$. That is, for each  $b \in B$ there is no more than one $a \in A$
such that $f(a) = b$; but there may be zero!\\

\define If $h$ is a relation on $A,B$ and $k$ is a relation on $B,C$, then we
can create a relation $k \circ h$, called the \index{composition} {\em
composition} of $h$ and $k$, on $A,C$ defined by
\begin{displaymath}
(a,c) \in k \circ h \iff \exst b \in B \st (a,b) \in h \land (b,c) \in k
\end{displaymath}
So $k \circ h$ relates $a$ to $c$ by way of $b$.

If $f:A\rightarrow B$ is a function and $g:B\rightarrow C$ is a function, then
$g \circ f:A\rightarrow C$ is a function, defined as above, since all
functions are relations.\\

\define A relation $\mathcal{R}$ on $A,A$ is \index{transitive} {\em transitive}
if $(a\ \mathcal{R}\ b) \land (b\ \mathcal{R}\ c) \Rightarrow a\ \mathcal{R}\ c$ for all $a,b,c \in A$.

This is equivalent to saying $\mathcal{R} \circ \mathcal{R} = \mathcal{R}$.\\

\define A relation $\mathcal{R}$ on $A,A$ is \index{symmetric} {\em symmetric}
if $(a\ \mathcal{R}\ b) \Rightarrow (b\ \mathcal{R}\ a)$ for all $a,b \in A$.\\

\define A relation $\mathcal{R}$ on $A,A$ is \index{reflexive} {\em reflexive} 
if $a\ \mathcal{R}\ a$ for all $a \in A$.\\

\define A relation $\mathcal{R}$ on $A,A$ is \index{antisymmetric} {\em
antisymmetric} if $(a\ \mathcal{R}\ b) \land (b\ \mathcal{R}\ a) \Rightarrow
(a = b)$ for all $a,b \in A$.\\

An example of a relation which is symmetric, but not transitive is $\neq$. A
relation that is transitive, but not symmetric is $\leq$.\\

Let $A$ be a set and $\mathcal{R}$ any binary relation on $A$. Consider the
collection of all 
binary relations on $A$ which contain $\mathcal{R}$ and are transitive. First,
note that $A \times A$ is such a relation, so this set is nonempty. Call this
collection $\mathcal{C}$.

Now, consider $\mathcal{T} = \intover{C}{\mathcal{C}}C$. Clearly, $\mathcal{R}\
\subseteq \mathcal{T}$, since $\mathcal{R}\ \subseteq C$ 
for all $C \in \mathcal{C}$.
Also, $\mathcal{T}$ is transitive: If $(a,b)$ and $(b,c)$ are
in $\mathcal{T}$, then they are in
each $C \in \mathcal{C}$; all of which are transitive and so contain $(a,c)$.
And so, $(a,c)$ is also in $\mathcal{T}$. So $\mathcal{T}$ is transitive.

Furthermore, for all $C \in \mathcal{C}$,
$\mathcal{T} \subseteq C$. So $\mathcal{T}$ is the
smallest binary operation on $A$ that contains $\mathcal{R}$ and is
transitive. We call $\mathcal{T}$ the \index{closure!transitive} {\em
transitive closure} of $\mathcal{R}$.

Notice that we haven't exactly established what $\mathcal{T}$ really is.
Still, it is useful to know that it exists; and with some effort we can do
this. We can construct the \index{closure!symmetric} {\em symmetric} and
\index{closure!reflexive} {\em reflexive} closure in a similar fashion. We can
even  construct the reflexive, symmetric, and transitive closure if we so
desire.\\

\define A binary relation on a set $A$ is an \index{relation!equivalence} {\em
equivalence relation} if it is reflexive, symmetric, and transitive.\\

The obvious example of an equivalence relation is =, while
congruence module $x$, for any $x \in \integers$, would be another.\\


\define A binary relation ``$\leq$'' on a set $A$ is a \index{order!partial}
{\em partial order} if it is reflexive, transitive, and antisymmetric.\\

There are two important partial orders with which we are already well
acquainted. In $\reals$, $\leq$ is a partial order. Given some 
set $X$, $\subseteq$ is a partial order in $P(X)$.\\

\define If $\leq$ is a partial order on $A$, then we say it is a 
\index{order!total} {\em total order} if 
$\fall a,b \in A,\ (a \leq b) \lor (b \leq a)$.\\

\define A function $f:A\rightarrow B$ is a \index{one to one correspondence}
{\em one to one correspondence} if it is 1 -- 1 and onto.

If there is a 1 -- 1 correspondence between two sets $A$ and $B$, we write
$A \sim B$; and we can say that $A$ and $B$ have the same size.\\

The fact that a composition of 1 -- 1 correspondences is a 1 -- 1
correspondence follows directly from the facts that a composition of 1 -- 1
functions is 1 -- 1, and a composition of onto functions is onto.\\

\define If $\mathcal{R}$ is a binary relation, then $\mathcal{R}^{-1}$ is
defines by $[(a,b) \in\ \mathcal{R}^{-1} \iff (b,a) \in \mathcal{R}]$.\\

\exercise For a function $f$, $f$ is a 1 -- 1 correspondence if and only if
$f^{-1}$ is a function.\\

\exercise $\sim$ is an equivalence relation.\\

Clearly $\emptyset \not \sim \{\emptyset\}$, $\{\emptyset\} \not \sim
\{\emptyset,\{\emptyset\}\}$, and for any finite set $A$, $\naturals \not \sim
A$. But what about $A \sim B$ when $A$ and $B$ are infinite? Well, obviously
there are infinite sets such that this relation holds, since we can find some
which have a 1 -- 1 correspondence; but in fact it is not true for all
infinite sets.

\subsection{Cantor's Diagonal Argument}
\index{Cantor}
\index{Diagonal Argument}

\begin{theorem} 
There is no 1 -- 1 correspondence between a set $X$ and it's power set $P(X)$.
More precisely, for any function $f:X\rightarrow P(X)$, $f$ is not onto.
\end{theorem}

\begin{proof}
Given such a function $f$, consider the predicate $S(x) := (x \nin f(x))$. 
Notice that since
$f:X\rightarrow P(X)$, $f(x) \subseteq X$, and for any $x \in X$, it actually
makes sense to consider $S(x)$. Call the class defined by formation with
$S(x)$ $\mathcal{C}$. Since $\fall x \in \mathcal{C},\ x \in X$, we can use
sifting (with $X$) to show that $C \subseteq X$. So $\mathcal{C} \in P(X)$. It
is, in fact, the set of all elements $x$ of $X$ for which $f(x)$ does not
contain $x$.

We will now show that there is no $x \in X$ such that $f(x) = \mathcal{C}$.
For an arbitrary $x \in X$. we have two cases: either $f$ is in $\mathcal{C}$
or it is not. In the first case, then, $f(x) \neq \mathcal{C}$, since if it
did, $x$ would not be in $C$. In the second case, $f(x) \neq \mathcal{C}$
because $x \in f(x)$ and $x \nin \mathcal{C}$.

To put it another way: we have constructed this set $\mathcal{C}$ in such a
way that the following holds: 
\begin{displaymath}
\fall x \in X,\ f(x) = \mathcal{C} \Rightarrow x \in f(x) \land x \nin
(\mathcal{C} = f(x)).
\end{displaymath}
Clearly this is a contradiction. Hence, for every $x \in X,\ f(x) \neq
\mathcal{C}$. 

We have found a set in $P(X)$ to which $f$ cannot map any element of $X$.
Therefore, $f$ cannot be onto. Since $f$ and $X$ were arbitrarily chosen, we
see that no function that maps a set onto its power set can be onto. This
means that there can be no 1 -- 1 correspondence between any set and its power
set.
\end{proof}

\define A set $X$ is said to be \index{countable} {\em countable} if 
there exists a 1 -- 1 correspondence between $X$ and $\naturals$, the natural
numbers.\\

Cantor's diagonalization argument can be used to prove that the real numbers,
$\reals$, are not countable. In computer science, it has been used to prove the
``halting problem'', which says that it is not possible to write a program
which takes as input another program and decides whether or not that program
will run forever.\\
\\

Generally, when speaking of infinite sets, because there can be no onto
function from a set to it's power set, we say that the power set is ``bigger''
than the set. Dr. Brin prefers the phrase ``more complex'' since discussion of
sizes of infinity is a bit of a can of worms.\\

\define For sets $X$ and $Y$, if there exists a function $f:X\rightarrow Y$
such that $f$ is a 1 -- 1 correspondence, then we say $X$ and $Y$ have the
same \index{cardinality} {\em cardinality}. We write {\tt cardinality} $X =$
{\tt cardinality} Y, or just $|X| = |Y|$.\\

\section{On Infinity \& $\naturals$}

Recall how we defined the natural numbers: $0 = \emptyset$, $1 =
\{\emptyset\} = \{0\}$, $2 = \{\emptyset, \{\emptyset\}\} = \{0,1\}$, $\cdots$.
Notice that for any number $n$, the set that it corresponds to contains the
$n$ elements, $0, ..., n - 1$.\\

\define We say a set is \index{finite} {\em finite} if it has 1 -- 1
correspondence with some $n \in \naturals$. If a set is not finite, then it is
infinite.\\

Let us assume that a set $X$ is infinite; i.e. it has no 1 -- 1 correspondence
with any $n \in \naturals$. Notice $X \neq \emptyset$ by this assumption; so
there is at least some $x_0 \in X$. We can define a function 
$f_1:\{0\}\rightarrow X$ by $f_1(0) = x_0$. Clearly, this $f_1$ is
1 -- 1, but not onto. 

Since $f_1$ is not onto, there is some $x_1 \in X$ with $x_1 \neq x_0$. So now
let the function $f_2:\{0,1\}\rightarrow X$ so that $f_2(0)=x_0$ and $f_2(1) =
x_1$. Again, we have a function which is 1 -- 1 and not onto.

Following this pattern, assume that we have $n$ such 1 -- 1 functions:
\begin{itemize}
\item[] $f_1:(\{0\} = 1) \rightarrow X$
\item[] $f_2:(\{0,1\} = 2) \rightarrow X$
\item[] $\vdots$
\item[] $f_n:(\{0,...,n-1\} = n) \rightarrow X$
\end{itemize}
so that $f_n(n)=x_{n-1}$ and for all $1 \leq i < j \leq n$ $f_i$ is a restriction
of $f_j$ to $\{0,...,i-1\}$. Of course, $f_n$ is also not onto. So there is
some $x_n \in X-($ Image of $f_n)$.

Now define $f_{n+1}(j)$ as follows: 
\begin{displaymath}
f_{n+1}(j) = \left\{ \begin{array}{ll}

f_n(j) & \textrm{if } j \leq n - 1\\
x_n & \textrm{if } j = n
\end{array} \right.
\end{displaymath}
Again, $f_{n+1}$ is 1 -- 1 and not onto. Also, for $i < j$, $f_i$ is still a
restriction of $f_j$ (though this is tedious to prove). By the
{\em principle of inductive definition} we can state that for any $n \in
\naturals$, there is an $f_n$.\\

Let $f = \unionover{n}{\naturals}f_n$. Remember, functions are really just
maps: sets of ordered pairs. We could also just write ``let $f(j) = f_n(j)$
for any $n > j$''. Since each $f_i$ is a restriction of $f_j$ when $i < j$,
$f$ is well-defined. 

We claim that $f$ is 1 -- 1. If $i \neq j$, take $m > \max(i,j)$. Then $f(i) =
f_m(i)$ and $f(j) = f_m(j)$; and since $f_m$ is 1 -- 1, $f_m(i) \neq f_m(j)$.
Since this works for any $i \neq j$, $f$ is 1 -- 1.

We have created, for an arbitrary infinite set $X$, a 1 -- 1 function from
$\naturals$ to $X$. This means that every infinite set is at least as large as
$\naturals$; so $\naturals$ are (in a sense) the smallest infinite set.\\

\subsection{Induction, Weak and Strong}

Recall that the natural numbers have the ``inductive'' property:
\begin{displaymath}
(A \subseteq \naturals) \land (0 \in A) \land (\fall n \in A,\ n + 1 \in A)
\Rightarrow A = \naturals.
\end{displaymath}

This is equivalent to stating that if a predicate $S(n)$ is valid for natural
numbers, and $S(0)$ is true, and for all $n$, $S(n) \Rightarrow S(n+1)$; then
for all $n \in \naturals, S(n)$.\\

Recall the successor function; and consider $s:\naturals \rightarrow
\naturals$. Clearly $s$ is 1 -- 1, and 0 is the only element not in the image
of $s$.\\

\define We define $+:\naturals \times \naturals \rightarrow \naturals$ as:
\begin{itemize}
\item[] $n + 0 = n$, and
\item[] $n + s(m) = s(n + m)$.
\end{itemize}
We can use induction to prove that $n+m$ is defined for all $n$ and $m$ in
$\naturals$. If I were industrious, I would have done so.\\

\define We define the relation $``\leq'' \subseteq \naturals \times \naturals$
as $n \leq m \iff \exst k \in \naturals \st m = n + k$.\\

\define A \index{least element} {\em least element} $x$ of $A$  must satisfy:
\begin{itemize}
\item[$(i)$] $x \in A$
\item[$(ii)$] $\fall x \in A,\ x \leq A$
\end{itemize}
\
\define If $X$ is a set and $\leq$ is a partial order on X, then we say $\leq$
is a \index{order!well} {\em well order} if for each nonempty subset subset 
$A \subseteq X$, there is a least element of $A$.\\

For example, $\leq$ is not a well order on $\reals$ or $\integers$, since
neither has a least element. It is, however, a well order on $\naturals$. \\

\exercise Prove that $\leq$ is a well order on $\naturals$.\\

Let $S(n)$ be a statement valid on $\naturals$ such that 
\begin{itemize}
\item[(1)] $S(0)$ is true and 
\item[(2)] $\fall n \in \naturals,\  [(\fall m < n,\ S(m)) \Rightarrow S(n)]$.
\end{itemize}
The reason for this predicate will become evident soon.

In trying to prove the statement $(\fall n,\ S(n))$,
we can let $F = \{n \in \naturals \mid
S(n) \textrm{ is false }\}$ and try to show that $F = \emptyset$. If $F$ is
nonempty, then it has a least element, $j$, since $\naturals$ is 
well ordered and $F \subseteq \naturals$. This $j$ is the smallest natural
number such that $S(j)$ is false. So $\fall k < j,\ S(k)$ is true. But then
$j$ cannot exist: $j \neq 0$ by condition (1), and $j$ cannot be anything
else, by condition (2). Therefore $F = \emptyset$ and 
$\fall n \in \naturals,\ S(n)$.\\

We have just found another form of induction. Let us compare the two forms we
have:
\begin{equation}
(S(0) \land \fall k > 0 \in \naturals,\ [(\fall m < k,\ S(0))
\Rightarrow S(k)]) \Rightarrow \fall n \in N,\ S(n) \label{strong}
\end{equation}
and
\begin{equation}
(S(0) \land [\fall k > 0 \in \naturals,\ S(k-1) \Rightarrow S(k)]) \Rightarrow
\fall n \in \naturals S(n) \label{weak}
\end{equation}

Recall from the section on logic (which I may or may not have 
actually typed up) that we can evaluate logical implications in the hopes of
determining their relative usefulness. 

Here we have two statements with the same conclusion. So let us look at the
hypotheses:\\

\ref{strong}: 
$(S(0) \land \fall k > 0 \in \naturals,
\ [(\fall m < k,\ S(m)) \Rightarrow S(k)])$
\\

\ref{weak}: 
$(S(0) \land [\fall k > 0 \in \naturals,\ S(k-1) \Rightarrow S(k)])$\\

Since they both have the expression $S(0) \land \fall k > 0 \in \naturals,)$,
we can ignore that and look at what remains:\\

\ref{strong}: $[(\fall m < k,\ S(m)) \Rightarrow S(k)]$\\

\ref{weak}: $[S(k-1) \Rightarrow S(k)]$\\

Again, we have two implications with the same conclusion; so look further:\\

\ref{strong}: $(\fall m < k,\ S(m))$\\

\ref{weak}: $S(k-1)$\\

We can say that $\fall m < k,\ S(m)$ a stronger statement than $S(k-1)$ since
it is a broader statement. Recall that a stronger hypothesis makes a weaker
implication. So the statement\\

$(S(0) \land [\fall k > 0 \in \naturals,\ S(k-1) \Rightarrow S(k)])$\\

is stronger than \\

$(S(0) \land \fall k > 0 \in \naturals,
\ [(\fall m < k,\ S(m)) \Rightarrow S(k)]).$
\\

But then, again, we are still in a hypothesis; so statement \ref{strong} is
stronger than \ref{weak}. This is why statement \ref{strong} is known as the
\index{induction!strong} {\em strong form of indunction} and statement
\ref{weak} \index{induction!weak} the {\em weak form}. \\

The point is: would you rather prove a statement based on the assumption that
$\fall m < k,\ S(k)$ or just that $S(k - 1)$ is true?\\

A practical example of this is to prove that every $j \in \naturals$ is a
product of primes. We define $S(j) := (j = 0 \lor j = 1 \lor j \textrm{is a
product of primes})$.

We want to prove $S(n)$, assuming $\fall j < n,\ S(j)$. There are two cases.
In the first case, $n$ is a prime number, and we are done. In the second case,
$n$ is not a prime number, so $n = a \dot b$ for some $a < n$ and $b < n$.
But we know that $S(a)$ and $S(b)$ hold true. So $a$ and $b$ are products of
primes, and therefore $n = a \dot b$ is also a product of primes.

This proof could be accomplished with weak induction, but it would be a major
pain. Most likely, one would have to essentially prove strong induction via
weak induction to do so.\\

Practically speaking, strong induction is much more useful than weak
induction. But, in fact, they are logically equivalent.



































\chapter{Real Numbers}

It is good to understand the properties of the real numbers. It is.\\

\index{field}
\define A field is a set containing the following operators:
$0,\, 1,\, +,\, -,\, \cdot,$ and $^{(-1)} $. 

It may be strange to think of $0$ and $1$ as operators, but you can think of
them as operators of rank 0; i.e. they require no arguments. Note that the $-$
is rank 1; that is, it is the unary additive inverse operator. Also note that the multiplicative inverse, $^{(-1)}$ operates only on $\reals - \{0\}$.
\begin{enumerate}
\item $\naturals$ denotes the natural numbers
\item $\integers$ denotes the integers
\item $\rationals$ denotes the rational numbers
\item $\irrationals$ denotes the irrational numbers
\item $\reals$ denotes the real numbers
\end{enumerate}
The field must be
closed under addition and multiplication, and must satisfy the following
properties:
\begin{enumerate}
\item $a + b = b + a$ (commutativity, additive)
\item $\fall a, \exst -a \st a + (-a) = 0$ (additive inverse)
\item $a + (b + c) = (a + b) + c$ (associativity, additive)
\item $a + 0 = a$ (additive identity)
\item $a \cdot b = b \cdot a$ (commutativity, multiplicative)
\item $a \cdot (b \cdot c) = (a \cdot b) \cdot c)$ (associativity,
multiplicative)
\item $a \cdot a^{-1} = 1 $ for $a \neq 0$ (multiplicative inverse)
\item $a \cdot 1 = a$ (multiplicative identity)
\item $a \cdot (b + c) = a \cdot b + a \cdot c$ (distributivity)
\end{enumerate}

The set of real numbers, $\reals$, is a {\em field}. Furthermore, it is {\em
totally ordered} field. There is a binary operation $(\leq)$ on $\reals \times
\reals$ that is a total (or linear) order, satisfying 

$(a)\ a \leq b \Rightarrow a + c \leq b + c$ and 
$(b)\ $If $a \geq 0$ and $b \leq c$, then $a \cdot b \leq a \cdot c$.\\

Question: Does the above guarantee that $0 < 1$? Find out.\\

We can embed $\naturals$ in $\reals$ inductively by $0 \rightarrow 0$,
$\textrm{s(n)} \rightarrow n+1$. With this, we can think of $\naturals
\subseteq \reals$. (This is necessary because of how we previously defined the
natural numbers, which I haven't actually added to these notes yet.)\\

\index{Archimedean property}
The real numbers have the {\em Archimedean property}: $\fall x \in \reals
\exst n \in \naturals \st x \leq n$.\\

\index{least upper bound property}
\index{complete}
Finally, $\reals$ is {\em complete}: it has the least upper bound property. If
$S \subseteq \reals$ and $S$ has an upperbound, then the set of upper bounds
\begin{enumerate}
\item $\naturals$ denotes the natural numbers
\item $\integers$ denotes the integers
\item $\rationals$ denotes the rational numbers
\item $\irrationals$ denotes the irrational numbers
\item $\reals$ denotes the real numbers
\end{enumerate}
of $S$ has a least element.\\

Let some $C$ satisfy the conditions stated above; then there is some function
$f: C \rightarrow \reals$, a 1 -- 1 correspondence preserving those properties.
I.e. $f(0) = 0,\ f(-a) = -f(a),$ etc.\\







\chapter{Topology}

\section{Topologies}

In the same way that linear algebra is a nice general setting for discussing
linear combinations, topology is a nice general setting for discussion of 
{\em limits}, {\em convergence}, and {\em continuity}. We begin with the
definition stated earlier, which is the most typical way a topology is
defined:\\
\\
\define A {\em topological space} \index{topological space} is a pair $(X, T)$ where $X$ is a set and
$T$ is a collection of subsets of $X$ satisfying: 
\begin{enumerate}
\item $\emptyset \in T$.
\item $X \in T$.
\item $A, B \in T \Rightarrow A\cap B \in T$.
\item $A_{\alpha}, \alpha \in I, \in T \Rightarrow \bigcup A_{\alpha} \in T$.
\end{enumerate}

We call $X$ the {\em underlying set} and $T$ the {\em topology}. \\
\index{topology}
\indent
Often, the topological space (top. space) $(X, T)$ is referred to simply as
$X$. It is important to notice that this is 
{\em (a)} wrong,
{\em (b)} only done if there is a single topology being 
discussed in association with $X$, and {\em (c)} done all the time.\\
\\

\example Consider $\reals$, the set of real numbers, with 
\index{topology!usual on $\reals$}
\begin{equation}
T = \{S\subseteq
\reals \mid \forall\, x \in S \ \exists\, \varepsilon > 0 \st
(x-\varepsilon, x + \varepsilon) \subseteq S \}.
\end{equation}

$(\reals, T)$ is a topological space:
Clearly $T \subseteq P(\reals)$.
That $\emptyset \in T$ is vacuously true. That $\reals \in T$ is also trivial.
\\
\indent Second, given $\{A_{\alpha}\}, \alpha \in I$, a family of sets indexed over
set $I$ such that $\forall\, \alpha \in I, A_{\alpha} \in T$, we must have that 
$\bigcup_{\alpha \in I}\ A_{\alpha} \in T$; and we do: Let $W =
\bigcup_{\alpha \in I}\ A_{\alpha}$. 
Then $\forall\,x \in W, \exists\,\alpha \in I$
s.t. $x \in A_{\alpha}$; and so, by hypothesis $\exists\,\varepsilon > 0$ s.t.
$(x-\varepsilon, x+\varepsilon) \subseteq A_{\alpha} \subseteq W$.
\\
\indent Finally, given $A, B \in T$, we must have that $A \cap B \in T$. Let $x
\in A \cap B$ be given. Then, $\exists\, \varepsilon_{\alpha} > 0$ s.t. 
$(x-\varepsilon_{\alpha},x+\varepsilon_{\alpha})\subseteq A$ and $\exists\, \varepsilon_{\beta}$ 
s.t. $(x-\varepsilon_{\beta}, x+\varepsilon_{\beta}) \subseteq B$. Choose
$\varepsilon = min\{\varepsilon_{\alpha},\varepsilon_{\beta}\}$. Then
$(x-\varepsilon, x+\varepsilon) \subseteq A$ and 
$(x-\varepsilon, x+\varepsilon) \subseteq B$; and so
$(x-\varepsilon, x+\varepsilon) \subseteq A \cap B$. 
Therefore, $A\cap B \in T$. 

Hence, $T$ is a topology; it is called the ``usual topology on $\reals$''. \\

\claim If $T_1$ and $T_2$ are both topologies on $X$, then $T_1 \cap T_2$ is
at topology on $X$. Furthermore, if $T_\alpha$, $\alpha \in I$, is a family of
topologies on $X$, then $\intover{\alpha}{I}T_{\alpha}$ is a topology on
$X$. These are simple to verify by checking the conditions.\\
\\

\define Any element of $T$ in the topological space $(X,T)$ is called an
{\em open set}\index{set!open} of $(X,T)$ (with respect to $T$).
\\

\example There are always at least two topologies on any set $X$. The power
set of $X$, $P(X)$, is a topology on $X$, since everything is in it. This is
called the \index{topology!discrete} {\em discrete topology}. Also,
$\{\emptyset, X\}$ is a topology on $X$. This is called the
\index{topology!indiscrete} {\em indiscrete topology}. It is easy to verify
that these are topologies.

Why the word ``discrete''? Notice that in the discrete topology, every
singleton is open. The open singleton set $\{x\}$ for a given $x \in X$,
is interpreted to mean that $x$ (and nothing else!) is close to $x$. Hence, it
is discrete. \\

One should be careful when using the word ``open''. Open intervals in
$\reals$ are generally not the same thing as open sets in a topological
space. Open sets are a generalization of the concept of open intervals. Often,
they are denoted as $U$.

The idea of open sets represents ``all of the points near $x$'', but of
course, each open set containing $x$ may only contain some of the points near
$x$, since $x$ is likely to be contained in many distinct open sets. So,
rather, each open set containing $x$ represents one picture, or idea, of
points close to $x$.  We can only completely understand what is happening very
near $x$ when we look at {\em all} open sets containing $x$. 

We often refer to an open set containing $x$ as an ``open set about $x$'' or
an ``open neighborhood of $x$''.\\
\\
\begin{lemma}\label{unionlemma}
Let $A$ be a set, and $B \subseteq P(A)$. Let $C \subseteq A$. Then $C$ is a
union of elements of $B$ if and only if $\forall\, x \in C, \exists\, Q \in B$
s.t. $x \in Q$ and $Q \subseteq C$.
\end{lemma}

\begin{proof}
The statement ``$C$ is a union of elements of
$B$'' can be rewritten more formally as 
\begin{displaymath}
\exst\{D_{\alpha}\},\alpha \in I  \st
C = \bigcup_{\alpha \in I} D_{\alpha}
\textrm{ and  each }D_{\alpha} \in B.
\end{displaymath}

$(\Rightarrow)$ $\fall x \in C, \exst \beta \in I \st x \in D_{\beta}$,
since $C \subseteq \bigcup_{\alpha \in I} D_{\alpha}$. But $D_{\beta} \subseteq
\bigcup_{\alpha \in I} D_{\alpha} \subseteq C$. So let $Q = D_{\beta}$. Then 
$\fall x \in C,\; \exst Q \in B \st x \in Q \subseteq C$.

$(\Leftarrow)$ We are given the hypothesis: $\fall x \in C,\; \exst Q_{x} \in B$
s.t. $x \in Q_{x} \subseteq C$. So $C$ is itself the indexing set. I.e. 
$\exst \{Q_{x}\}, x \in I$, a family of sets indexed over $C$, such that 
$C = \bigcup_{x \in C} Q_{x}$ and each $Q_{x} \in B$. So $C$ is a union of all
such $Q_{x}$.
\end{proof}

As an application of this theorem, we can rewrite $T$ as 
$T = \{ S \subseteq \reals \mid S$ 
is a union of open intervals in $\reals\}$.
\\

\exercise Prove this by showing\\
$\{S \subseteq \reals \mid \fall x \in S \exst \varepsilon > 0$ s.t.
$(x - \varepsilon, x+\varepsilon) \subseteq S\} = $\\
$\{S \subseteq \reals \mid\ S$ is a union of open intervals in 
$\reals\}$ by mutual containment {\em (in notes of 10/18)}.\\

\section{Limits and Continuity}

\claim $\bigcup_{\alpha \in \emptyset} D_{\alpha} = \emptyset$. 

\begin{proof}From the definition of union, 
$x \in \unionover{\alpha}{\emptyset} D_{\alpha} \Leftrightarrow \exst \alpha \in
\emptyset \st x \in D_{\alpha}$. Since no such $\alpha$ can exist, there can be
no such $x$.
\end{proof}

\define Let $f:\reals \rightarrow \reals$ be a function. We say $f$ is {\em
continuous at $x$} if $\fall \varepsilon >0 \ \exst \delta > 0$ s.t. 
$y \in (x - \delta, x + \delta) \Rightarrow f(y) \in (f(x) - \varepsilon, f(x)
+ \varepsilon)$. \\
\\
\define If $\fall x \in \reals$, $f$ is continuous at $x$, then we say $f$ is 
{\em continuous on $\reals$}, or just {\em continuous}.

\begin{theorem}
If $f:\reals \rightarrow \reals$ is a function and $\reals$ is given the usual
topology $T$, then $f$ is continuous if and only if $\fall U \in T,\; f^{-1}(U)
\in T$.
\end{theorem}

\exercise prove this (it's in the handwritten notes)\\
\\
Based on this, we can redefine continuity in a more general way:\\
\\
\define Let $(X,T)$ and $(Y,S)$ be topological spaces. Let $f:X \rightarrow Y$
be a function. We say $f$ is {\em continuous} if $\fall A \in S,\; f^{-1}(A)
\in T$. \index{function!continuous}

I.e. $f$ is continuous if the inverse image of every open set is open.
\\

\define Let $(X,T)$ be a topological space, and let $x \in X$ and $W
\subseteq X$. We say $x$ is a {\em limit point}\index{limit point} of $W$ if
$\fall U \st x \in U$, $U \cap (W - x) \neq \emptyset$.\\

\define For a topological space $(X,T)$ with $W \subseteq X$ and $x \in W$, we
say $x$ is \index{isolated} {\em isolated in } $W$ if $x$ is not a limit point
of $W$.\\

\define For a topological space $(X,T)$ with $x \in X$, we say $x$ is {\em
isolated} if $x$ is isolated in X.\\

\example Consider $\reals$ with the usual topology, and let $W = (0,1)$. Is 0
a limit point of $W$? Yes. Any open interval containing 0 must contain points
in $W$. Is $\frac{1}{2}$ a limit point of $W$? Yes, for the same reason; note
that the fact that $\frac{1}{2} \in W$ is irrelevant. Is $-\frac{1}{2}$ a
limit point of $W$? No; many open intervals containining $-\frac{1}{2}$ will
not contain any points in $(0,1)$; for example, $(-1,0)$ does not.\\
\\
Recall that the usual topology on $\reals$ is the set of all unions of open
intervals in $\reals$. So, in a sense, this topology is generated by the open
intervals in $\reals$. We generalize this concept with bases.\\


\section{Bases}
\define Let $(X,T)$ be a topological space, and let $B \subseteq P(X)$. We say
$B$ is a {\em basis} \index{basis} for $T$ if $T = \{$ all unions of subsets
of $B$ \}.

Every topological space has a basis. In the very least, any topology is a
basis for itself.  \\
\\
If $B$ is a basis for $T$, then every element of $T$ is a union over subsets
of $B$. I.e. $T = \{ \unionover{S}{C}S\mid C \subseteq B \}$.\\
\\
\example In the $\reals$, the open intervals form a basis of the usual
topology.\\
\\
Let $C = \{1, \frac{1}{2}, \frac{1}{3}, \dots, 0\}$, and 
\begin{displaymath}
E = \reals - C = (-\infty, 0) \cup (1, \infty) \cup (\frac{1}{2}, 1) \cup 
(\frac{1}{3}, \frac{1}{2}) \cup \dots
\end{displaymath}

Is $E$ an open interval? No. Is it an open set in the usual topology on
$\reals$? Yes, because it is a union of open intervals.\\
\\

\claim If $B$ is a basis for topology $T$, then $B \subseteq T$: 
$U \in B \Rightarrow \{U\} \subseteq B$, and the union over $\{U\}$ is simply
$U$, since it is the only one there. Hence, $U \in T$.\\
\\
\exercise Let $(X,T)$ be a topological space, and $B$ a basis for $T$. Let 
$x \in X$ and $W \subseteq X$. Then $x$ is a limit point of $W$ if and only if 
$\fall U \in B$ with $x \in U$, $(U\cap(W - x) \neq \emptyset)$. \\
\\
\exercise Let $f:(X,T)\rightarrow(Y,S)$ be a function between topological 
spaces. Let $B$ be a basis for the topology $S$. Then $f$ is continuous if 
and only if $\fall U \in B, f^{-1}(U) \in T$.\\
\\

\example Consider function $f: \reals \rightarrow \reals$, with the usual
topology. $f$ is continuous if $f^{-1}((a,b))$ is open, where $(a,b)$ is any
open interval. It should be noted that we only require that $f^{-1}((a,b))$ be
in the usual topology (i.e. a union of open intervals), not necessarily an
open interval itself. That would be too restrictive: consider $f(x) = x^2$;
which is obviously continuous. 
But $f^{-1}((1,2)) = (-\sqrt{2}, -1) \cup (1, \sqrt{2})$, which is not a basis
element.\\
\\
\begin{theorem}
\label{basisthm}
Let $X$ be a set, and $B \subseteq P(X)$. $B$ is 
a basis for a topology on $X$ (i.e. $\{\unionover{S}{C}S\mid C \in B\}$ a 
topology on $X$) if and only if 

\begin{enumerate}
\item $\unionover{S}{B} S = X$ and 
\item $A,\;B \in B \Rightarrow A \cap B$ is a union of some elements of $B$.
\end{enumerate}
\end{theorem}

\begin{proof}
\noindent
$(\Rightarrow)$ Assume there is a topology $T$ on $X$ for which $B$ is a
basis.

$1)$ We must show that $\unionover{S}{B}S = X$:
\\[2pt]

\noindent
$(\subseteq): B \subseteq P(X) \textrm{, so } \fall S \in B, \fall x \in S, 
x \in X, so \unionover{S}{B} S \subseteq X.$
\\[2pt]

\noindent
$(\supseteq): T \subseteq \{\unionover{S}{C} S\mid C \subseteq B\}$ ,
but $ X \in T \textrm{. So } \exst C^{\prime} \subseteq B \st X =
\unionover{S^{\prime}}{C^{\prime}}S^{\prime}$. Hence, $ \fall x \in X,
x \in \unionover{S^{\prime}}{C^{\prime}}S^{\prime}$.

But $ \unionover{S^{\prime}}{C^{\prime}}S^{\prime} \subseteq \unionover{S}{B}S$
: $\fall S^{\prime} \in C^{\prime}$, $S^{\prime} \in B$, so 
$\fall y \in \unionover{S^{\prime}}{C^{\prime}}S^{\prime}$, $y \in 
\unionover{S}{B}S$. So $\fall x \in X$, $x \in \unionover{S}{B}S$. Hence, 
$X \subseteq \unionover{S}{B}S.$\\[2pt]

$2)$ We must show that $A,A^\prime \in B \Rightarrow \fall x \in A \cap
A^\prime$, 
$\exst C \in B \st x \in C \subseteq A \cap B$.

Assume $A$ and $A^\prime$ are in $B$. Then $A \cap A^\prime \in T$, since $T$ is closed
under pairwise intersections.  Then $A \cap A^\prime$, being in $T$, is a union of
elements of $B$, because $B$ is a basis of $T$. We have previously shown that
a $F$ is a union of elements of a set $G$ if and only if $\fall x \in F$, 
$\exst H \in G \st x \in H \subseteq F$. So we are done.
\\[2pt]

\noindent
$(\Leftarrow)$ Given conditions $(1)$ and $(2)$, let $T$ be the set of all
unions of elements of $B$. We must show that $T$ is topology on $X$. 
\\[1pt]

\noindent
$\emptyset \in T$: $\emptyset$ is a union of 0 elements of $B$, 
so this is true.
\\[1pt]

\noindent
$X \in T$: $\unionover{S}{B}S = X$ by condition $(1)$, so $X$ is the union of
all elements of $B$. So this is true.
\\

\noindent
$A,A^\prime \in T \Rightarrow A \cap A^\prime \in T$:

In other words, we need to show that $A \cap A^\prime$ is a union of elements of $B$. We know that 
$\fall x \in A \exst G \in B \st x \in G \subseteq A$ and 
$\fall y \in A^\prime \exst H \in B \st y \in H \subseteq A^\prime$. Given 
$w \in A \cap A\prime $, then, $\exst G (\in B) \subseteq A \st w \in G$ and 
$\exst G^\prime (\in B) \subseteq A^\prime \st w \in G^\prime$. So, by
condition $(2)$, $\fall w \in G \cap G^\prime$, $\exst F \in B \st w \in F
\subseteq (G \cap G^\prime) (\subseteq (A \cap A^\prime)) $. By
Lemma \ref{unionlemma}, the proof is complete.
\end{proof}

\noindent
$A_{\alpha}, \alpha \in I, \in T \Rightarrow \bigcup A_{\alpha} \in T $:

Any element of $T$ is a union of elements of $B$; so
$\unionover{\alpha}{I}A_{\alpha}$ is a union of unions of elements of $B$. We 
will show that, in general, 
any union of unions of elements of some set is a union of elements
of that set, which is really quite obvious, but what the hell:

Given a family of sets which are unions of elements of some 
$G \subseteq P(Q)$, where Q is a set, i.e.
$\{Y_\alpha\} = \{\unionover{S}{C_\alpha}S \mid C_\alpha \subseteq G\}$, 
for $\alpha \in I$, some indexing set, we must show that 
$\fall \psi \in \unionover{\alpha}{I}Y_\alpha,\ \exst F \in G \st \psi \in F
\subseteq \unionover{\alpha}{I}Y_\alpha$.

Take $\psi \in \unionover{\alpha}{I}Y_\alpha$. 
$\exst \beta \in I \st \psi \in Y_{\beta}$. Now, 
$Y_{\beta} = \unionover{S}{C_\beta}S$, for some
$C_\beta \in G$, so $\exst S \in C_\beta$, call it $S_\psi$, such that 
$\psi \in S_\psi$ and $S_\psi \subseteq Y_\beta \subseteq
\unionover{\alpha}{I}Y_\alpha$. Furthermore, $S_\psi \in G$, since $S_\psi \in
C_\beta \subseteq G$.

Hence, $\fall \psi \in \unionover{\alpha}{I}Y_\alpha,\ \exst S_\psi \st \psi
\in S_\psi \in G$. So the union of unions of elements of a set is itself a
\claim If $T_1$ and $T_2$ are both topologies on $X$, then $T_1 \cap T_2$ is
at topology on $X$. Furthermore, if $T_\alpha$, $\alpha \in I$, is a family of
topologies on $X$, then $\bigcap_{\alpha \in I}T_{\alpha}$ is a topology on
$X$. These are simple to verify by checking the conditions.\\
\\
union of elements of that set. 

This proves that any union of elements of $T$ is itself a union of elements of
$B$, and is therefore in $T$; which complete the demonstration of the
conditions required to make $T$ a topology.

\exercise For a topological space $(X,T)$, the following are equivalent
(TFAE):
\begin{enumerate}
\item $T$ is the discrete topology
\item Every $x \in X$ is isolated
\item Every $\{x\}$ is open
\item $B = \{ \{x\} \} \mid x \in X \}$ is a basis for T
\end{enumerate}

The solution is in the hand-written notes.
\\

\exercise Let $B = \{[a,b) \mid a < b$ and $a,b \in \reals\}$. Show that $B$ is
a basis for a topology on $\reals$. This topology is called the
\index{topology!half-open} {\em half-open topology} on $\reals$.

Show that the usual topology on $\reals$ is a proper subset of the half-open
topology, which in turn is a proper subset of the discrete topology.

\section{Subbases}

Let $X$ be a set and $S \subseteq P(X)$. Let 
$F = \{T \subseteq P(X) \mid T$ is a topology on $X$ and $S \subseteq T\}$.
$F$ is not empty since it contains at least $P(X)$, the discrete topology.

Consider $T_S = \intover{T}{F}T$. Being an intersection of topologies, $T_S$
is itself a topology. Furthermore, $S \subseteq T_S$, since $S \subseteq T
\fall T \in F$. 

Therefore, $T_S$ must be the smallest topology on $X$ containing $S$; smallest
here meaning that for any topology $T$ on $X$ such that $S \subseteq T$, it
must be that $T_S \subseteq T$ also.\\
\\
\exercise For a topological space $(X,T)$, the following are equivalent
(TFAE):
\begin{enumerate}
\item $T$ is the discrete topology
\item Every $x \in X$ is isolated
\item Every $\{x\}$ is open
\item $B = \{ \{x\} \} \mid x \in X \}$ is a basis for T
\end{enumerate}
\define $T_S$ is called the topology generated by $S$, and $S$ is called a
\index{subbasis} {\em subbasis} of $T_S$. (Note: the phrase ``generated by''
has another, more common, meaning in topology).\\
\\
Let the  $B$ be the set containing all intersections of finitely many elements of
$S$. By convention, when taking an intersection of no subsets of an agreed
upon set $X$, the intersection is $X$ itself. By this convention, we are
guaranteed that $X \in B$. T$_2$

Notice that if $P,Q \in B$, then $P = H_1 \cap H_2 \cap \dots \cap H_r$ where
each $H_i \in S$ and $Q = K_1 \cap K_2 \cap \dots \cap K_t$ where each $K_i
\in S$; and $P \cap Q = H_1 \cap \dots \cap H_r \cap K_1 \cap \dots \cap K_t
\in B$, so $B$ is closed under pairwise intersection. 

By Theorem \ref{basisthm}, therefore, $B$ is a basis for some topology on $X$. Call
this topology $T_B$.\\ 

\claim $T_B = T_S$:

\begin{proof}
$(\supseteq)$: $S \subseteq B$ since $B$ contains intersections of single
elements from $S$, and $B \subseteq T_B$; so $T_B$ is a topology on $X$
containing $S$ and therefore $T_S \subseteq T_B$, since $T_S$ is the smallest
topology on $X$ containing $S$.\\

$(\subseteq)$: $T_S$ is a topology on $X$; so all finite intersections of $S$,
a.k.a. all elements of the set $B$, are contained in $T_S$. So $B \subseteq
T_S$. But $T_B$ is just all the unions of finite intersections of of subsets
of $B$, so since $T_S$ is closed under arbitrary union, $T_B \subseteq T_S$.
\end{proof}

\section{Sequences \& Convergence}

\subsection{Sequences}

\index{sequence}
\define A {\em sequence} in a set $X$ is a function from $\naturals$ to $X$.
For a sequence $a \in X$, we usually write $a_i$ rather than $a(i)$.\\

\index{convergence}
\index{limit!of a sequence}
\define If $(a_i)$ is a sequence in a topological space $(X,T)$, then $y \in
X$ is the {\em limit of } $(a_i)$ if $\fall \textrm{ open } U \textrm{ about }
y$, $\exst N \in \naturals \st (i > N) \Rightarrow (a_i \in U))$.

We say $\lim_{i \rightarrow \infty} a_i = y$, $a_i \rightarrow y$, or ``$a_i$
converges to $y$''.\\

\exercise In the above; if $T$ has a basis $B$, then $a_i \rightarrow y$ if
and only if $\fall$ basic open $U$ around $y$, $\exst N \in \naturals \st i >
N \Rightarrow a_i \in U$.  Is this true for the subbasis?\\

\define Given a topological space $(X,T)$ If $a:\naturals \rightarrow X$ is a
sequence then a \index{subsequence} {\em subsequence} is a composition
$a \circ j: \naturals \rightarrow X$ where $j$ is strictly increasing.
(I.e. $(i_1 > i_2 \Rightarrow j(i_1) > j(i_2)$. We denote the 
$a(j(i))$ as $(a_{j_i})$.


\subsection{Hausdorff Spaces}
\index{Hausdorff}
\index{T$_2$}
\define A topological space $(X,T)$ is {\em Hausdorff} (also called {\em \T{2}}) 
if $\fall x,y \in X, x \neq y,$ there exist open sets $U$ and $V$ such that $x
\in U$, $y \in V$, and $U \cap V = \emptyset$.

``In Hausdorff spaces, points really don't ever get too close together''. \\

\exercise For a Hausdorff topological space $(X, T)$, let $(a_i)$ be a
sequence in $X$. If $a_i \rightarrow x$ and $a_i \rightarrow y$, then $x = y$.

Proof in hand-written notes.\\

The above exercise implies that if $X$ is not Hausdorff, then it may be that
$a_i \rightarrow x$, $a_i \rightarrow y$, and $x \neq y$. Indeed, this is the
case. Consider $\reals$ with the indiscrete topology, and the sequence $a_i =
0$. $\fall x \in \reals$, $x$ is a limit point of $a_i$. After all, $\reals$
is the only open set containing $x$, and $\reals$ contains the entire
sequence.\\

Is $\reals$ with the usual topology Hausdorff? Yes.  For two points $x$
and $y \in \reals$, let $x < y$. Let $\delta = \frac{y - x}{2}$. Then $x \in
(x-\delta,x+\delta)$, and $y \in (y-\delta, y_\delta)$, and these two sets are
disjoint. 

How about $\reals$ with the discrete topology? If $x \neq y$, then $\{x\}$ and
$\{y\}$ satisfy the definition of \T{2}. (REMARK: I think I can argue that the
usual topology is Hausdorff, and it is a subset of the discrete topology, so
the discrete topology must also be Hausdorff.)

Obviously, the indiscrete topology is not Hausdorff, since for a given set $X$
with $x \neq y \in X$, the only set containing $x$ or $y$ is $X$, which
contains both.

How about $\reals$ with the half-open topology? Yes. If $x \neq y$, say $x <
y$, consider $[x,y)$ and $[y,y+\epsilon)$ for any $\epsilon > 0$.\\

\example Let $a_i = 1 - \frac{1}{2^i}$, i.e. $a_i = 0, \frac{1}{2},
\frac{3}{4}, \dots$.

\exercise Show that in the usual topology $a_i \rightarrow 1$. Notice that in
the half-open topology, $a_i \nrightarrow 1$ since $[1,2)$ is open about $1$,
but no $a_i$ is in $[1,2)$. In fact, it can be proven that in the half-open
topology, this sequence does not converge to any $x$ in $\reals$. Prove it.\\

Now, consider $a_i = 1 + \frac{1}{2^i}$, i.e. $a_i = 2, 1\frac{1}{2}, 
1\frac{1}{4}, \dots$

In the usual topology, $a_i \rightarrow 1$; this is also
true in the half-open topology.

\begin{proof}
We need to show that for any basic set $[p, q)$ containing $1$, there
is an $N \in \naturals \st i > N \Rightarrow a_i \in [p,q)$. 

Let $d = q - 1 > 0$. We want $\frac{1}{2^i} < d$, or $2^i > \frac{1}{d}$.
Since the logarithm is an increasing function, we can also express this as 
$\log_2 2^i > \log_2 \frac{1}{d}$, and $i > -\log_2 d$ by the properties of
logarithms.

So, for $N \in \naturals \st N > -\log_2 d,\ i > N \Rightarrow i > -\log_2 d
\Rightarrow \frac{1}{2^i} > d$ and $a_i = 1 + \frac{1}{2^i} < 1 + d = 1 + (q -
1) = q$. So $p \leq 1 < 1 + \frac{1}{2^i} < q$, thus $1 + \frac{1}{2^i} \in
[p,q)$.
\end{proof}

Furthermore, since we believe the half-open topology to be Hausdorff; $1$ must
be the only point to which this sequence converges.\\

Note that while it may seem reasonable that every limit point of a set $S$ is
a limit of a sequence in $S$, this is wrong. \\

\section{Closed Sets}

\index{closed}
\define For topological space $(X, T)$, we say $S$ is {\em closed in } $X$ if
it contains all of its limit points.  I.e. $S$ is closed in $X$ if and only
if:
\begin{displaymath}
y \textrm{ is a limit point of } S \Rightarrow y \in S.
\end{displaymath}

\claim For a topological space $(X,T)$ with $S \subseteq X$, if $a_i$
is a sequence in $S$ such that $a_i \rightarrow y$, then $y$ is in $S$
or $y$ is a limit point of $S$.

\begin{proof}
We will show that if $y \not \in S$, then $y$ is a limit point of $S$.
Let $U$ be open about $y$. Since $S - \{y\} = S$, we need only show that $U
\cap S \neq \emptyset$. By hypothesis, there is some $N \in \naturals \st 
i > N \Rightarrow a_i \in U$, but $a_i \in S \fall i$, so $a_i \in (S \cap
U)$. So $y$ is a limit point of $S$.

If $y$ is not a limit point of $S$; it must be an isolated point in $S$ (i.e.
it is possible that $y \in S$,  $a_i = y,\;y,\;y,\dots$ and there is some open
set containing $y$ which does not contain any other element of $S$). So it is
possible that a sequence contained entirely within a set converges to a point
which is not a limit point of that set; in which case, as we have just proven,
that point must be in the set.
\end{proof}

\corollary For topological space $(X,T)$, $S$ closed in $X$, $(a_i)$ a
sequence in $S$, with $a_i \rightarrow y$; it must be that $y \in S$. 

\begin{proof}
By above, either $y \in S$ or $y$ is a limit point of $S$. If $y$ is a
limit point of $S$, then $y \in S$, since $S$ is closed. 
\end{proof}

\claim In topological space $(X,T)$ with $A \subseteq X$, $A$ is closed if and
only if $X-A$ is open.

\begin{proof}
$(\Rightarrow)$ Take any closed set $A$, and and consider any $x \in X-A$.
Since $x \not in A$, $x$ is
not a limit point of $A$. So there is some open set $U$ about $X$ such that $U
\cap A - \{x\} = \emptyset$; so then $U \cap A = \emptyset$, which means $U
\subseteq (X - A)$. So $x \in U \subseteq (X - A)$. By Lemma
\ref{unionlemma}, then $(X-A)$ is a union of open sets and is therefore open. \\

$(\Leftarrow)$ If $X-A$ is open, then for any $x \in X-A$, $X-A$ is itself an
open set about $x$ which is disjoint from $A$. So no point in $X-A$ can be a
limit point of $A$. Therefore, all limit points of $A$ must be contained in
$A$, and so $A$ is closed.
\end{proof}

\corollary Given a topological space $(X,T)$, the set $\{X-U \mid U \in T\}$
is all the closed sets under $T$.

While it might seem to indicate that ``closed'' is the
opposite of ``open'', i.e. that for a given set the two conditions are mutually
exclusive, this is false! Indeed; since in a topological space $(X,T)$, the
complement of $X$ is $\emptyset$, and both are open; it must be that both are
also closed. (Which is trivial to see anyway).\\

In the discrete topology on a set $X$, every set is closed, since every set is
open. 

\example Consider $\reals$ with the usual topology. The following facts are
true:
\begin{enumerate}
\item Every singleton $\{x\}$ is closed, since its complement is a union of
open intervals, and therefore an open set.
\item Any closed interval is closed, for the same reason.
\item Half-open intervals ($[a,b)$ or $(a,b]$ where $a<b$) are neither open
nor closed.
\end{enumerate}

In summary, if you are trying to show that a certain set is closed (or open),
it is often a good idea to try showing that it's complement is open (or
closed); as in the following theorem.

\begin{theorem}
If a topological space $(X,T)$ is Hausdorff, then all singleton sets are
closed.
\end{theorem}

\begin{proof}
We show that $X - \{x\}$ is open. Let $y \in X - \{x\}$; i.e. $y \neq x$.
Since $X$ is Hausdorff, there are open sets $U$ and $V$ where $x \in U$ and $y
\in V$ and $U \cap V = \emptyset$, which means $V \subseteq X - \{x\}$.
We have now that for all $y \in X - {x}$
there is some $V \in T$ such that $y \in V \subseteq X - \{x\}$, which by
Lemma \ref{unionlemma} implies that $X - \{x\}$ is open.
\end{proof}

Note that the converse of the above theorem is not necessarily true.\\

\define A topological space $(X,T)$ where every proper subset of $X$ other
than $\emptyset$ is either open or closed, but not both is called a
\index{door space} {\em door space}.\\

\example One can generate a door space in the following way: choose $x \in X$
and let $T = \{S \subseteq X \mid x \in S\} \cup \{emptyset\}$. This is called
a \index{topology!principle} {\em principle topology}. Since a set (other than
$X$ or $\emptyset$) is open if and only if $x$ is in it, and $x$ is either
in a set or not, but not both; each set must be open or closed, but noth both.
\\

\begin{theorem}
Let $(X,T)$ be a topological space. Let $K = \{C \mid C$ closed in $X \}$.
Then
\begin{enumerate}
\item $\emptyset \in K$
\item $X \in K$
\item $K$ is closed under finite union.
\item $K$ is closed under arbitrary intersection.
\end{enumerate}
\end{theorem}

\begin{proof}
The first two statements are obvious; the final two follow from the fact that
$X-\unionover{\alpha}{I}A_\alpha = \intover{\alpha}{I}(X-A_\alpha)$ and
$X-\intover{\alpha}{I}=\unionover{\alpha}{I}(X-A_\alpha)$. The verification of
these statements is left as an easy exercise.
\end{proof}

Note the symmetry between $K$ and $T$. It could have been that topologies were
defined based on closed sets, and the open sets derived from them. This might
have been nice, actually, since intersections have a tendency to preserve
properties, which unions don't (for example, consider vector spaces). \\

\define Let $(X, T)$ be a topological space, with $A \subseteq X$. Consider
$\mathcal{C} = \{ C \subseteq X \mid A \subseteq C$ and $C$ closed $\}$.
$\mathcal{C}$ is nonempty since $X$ is in it. Let $\bar{A} =
\intover{C}{\mathcal{C}}C$; we call $\bar{A}$ \index{closure} the {\em
closure} of A.\\

$\bar{A}$ is the smallest closed set containing $A$, since every other closed
set containing $A$ contains $\bar{A}$. What is $\bar{A}$? We know that $A
\subseteq \bar{A}$, and the set of all limit points of $\bar{A}$ is contained
in $\bar{A}$, since it is closed.

Now consider $A^{?} = A \cup \{$ limit points of $A\}$. Is $A^{?}$ closed? \\

\exercise $P \subseteq Q \Rightarrow [x$ is a limit point of P $\Rightarrow
x$ is a limit point of $Q]$. \\

\exercise If $x$ is a limit point of $A^{?}$ then $x$ is a limit point of
$A$.\\

As a consequence of the above exercises, we see that $A^{?}$ is closed, and so
$\bar{A} = A^{?}$: We know that $A^{?} \subseteq \bar{A}$, and since $\bar{A}$
is the smallest closed set containing $A$, and $A \subseteq A^{?}$ and $A^{?}$
closed, we know that $\bar{A} \subseteq A^{?}$.\\

\exercise Prove that the following facts hold true in any topological space: 
\begin{enumerate}
\item $A \subseteq \bar{A}$
\item $\bar{\bar{A}} = \bar{A}$
\item $\bar{A \cap B} \subseteq \bar{A} \cap \bar{B}$
\item $\bar{A \cup B} = \bar{A} \cup B$ ?
\end{enumerate}

\example In $\reals$, with the usual topology:
\begin{enumerate}
\item $\bar{(0,1)} = \bar{[0,1)} = \bar{(0,1]} = [0,1]$
\item For $0 < x < 1$, $\bar{(0,x) \cup (x,1)} = [0,1]$
\item $\bar{(0,1) - \{\frac{1}{2},\frac{1}{3},\frac{1}{4}, \dots\}} = [0,1]$
\item $\bar{\rationals \cap (0,1)} = [0,1]$
\item $ \bar{\irrationals \cap (0,1)} = [0,1] $
\end{enumerate}

Notice that for sets $A$ and $B$, it does not necessarily hold that 
$\bar{A \cap B} = \bar{A} \cap \bar{B}$. For example, let 
$A = \rationals \cap (0,1)$ and $B = \irrationals \cap (0,1)$. Since $A \cap
B = \emptyset$, $\bar{A \cap B} = \emptyset$. But $\bar{A} \cap \bar{B} = [0,1]
\cap [0,1] = [0,1]$. 

Also, $\bar{\{x\}} = \{x\}\ \fall x \in \reals$, so
\begin{displaymath}
\bar{\unionover{x}{\rationals}\{x\}} = \bar{\rationals} = \reals \neq
\rationals = \unionover{x}{\rationals}\{x\} =
\unionover{x}{\rationals}\bar{x}.
\end{displaymath}

You can use closure to determine whether or not a set is closed, since $A =
\bar{A}$ if $A$ is closed. 

Also, you can think of closure as a function $\bar{}:P(X)\rightarrow P(X)$.
This implies that topologies can be approached from the point of view of
closures. For that matter, topologies may be approached from the point of view
of limit points, since closures are defined in terms of limit points.







\section{First Countability}

\define Given a point $x$ in a topological space $(X, T)$, a \index{neighborhood
base} {\em neighborhood base} for $x$ is a collection of open sets about $x$
such that every open set containing $x$ has at least one element of that
collection as a subset. That is, $\mathcal{N} \subseteq P(X)$ is a neighborhood base for
$x$ if and only if: 
\begin{displaymath}
\fall U \in T \st x \in U,\ \exst V \in \mathcal{N} \st V \subseteq U.
\end{displaymath}
Every point has at least one countable neighborhood base; namely, the set of
all open sets which contain that point.\\

\define A topological space $(X,T)$ is \index{first countable} {\em first
countable} if every $x \in X$ has a countable neighborhood base.\\

\example $\reals$, with the usual topology, is first countable.

Let $x \in \reals$ and 
$\mathcal{N} = \{(x - \frac{1}{n}, x + \frac{1}{n}) \mid n \in (\naturals - \{0\})\}$.
Then for a given open $U$ about $x$; there is $\varepsilon > 0,\ (x -
\varepsilon, x + \varepsilon) \subseteq U$.  So choose $(x - \frac{1}{n}, x +
\frac{1}{n})$ where $\frac{1}{n} < \varepsilon$. We know this exists because
the Archimedean property guarantees that for all $\delta > 0,\ \exst M \in
\naturals \st \frac{1}{M} > \delta$.\\

\begin{lemma}
\label{firstcountablesequencelemma}
Let $(X,T)$ be a first countable topological space. 
Let $S \subseteq X$ and $y$ a limit point of $S$. Then $y$ is a limit 
point of a sequence in $S$.
\end{lemma}

\begin{proof}
Let $\{U_1, U_2, \dots \}$ be a countable neighborhood base for $y$.
Let $V_1 = U_1,\ V_2 = U_1 \cap U_2, \dots$ so that $V_2 = V_1 \cap U_2$, 
$V_3 = V_2 \cap U_3$, and in general $V_{n + 1} = V_n + U_{n + 1}$.

Now, $V_1 \supset V_2 \supset V_3 \supset \dots$, and we have another
countable neighborhood base: $\{V_i \mid i > 1\}$. (We know that it is
countable since there is a 1 -- 1 correspondence with $\{U_i\}$, and we know it
is a neighborhood base since for any open $W$ about $y$, $\exst j \in
\naturals \st V_j \subseteq U_j  \subseteq W$.

Since $y$ is a limit point of $S$, and $y \in V_i$, we know that $V_i \cap (S
- \{y\}) \neq \emptyset$. For each $i \in \naturals$ use the axiom of choice to
choose some $a_i \in V_i \cap (S - \{y\})$. This creates a sequence $(a_i)$
which converges to $y$.  

Indeed, for any open $W$ about $y$, there is an $N \in \naturals \st V_N
\subseteq W$, and since for all $i > N, V_i \subseteq V_N$, we know that
$\fall i \geq N, a_i \in V_N \subseteq W$. Furthermore, each $a_i \in S$, so
the sequence $(a_i)$ is in $S$. 
\end{proof}
 
\section{T$_*$ Time (Separation Axioms)}

There is a set of categorize which help to further describe topological spaces
by analyzing properties related to open sets. They are called the 
\index{separation axioms} {\em separation axioms}, even though they are really
definitions. We have already seen one, the  

\define A topological space $(X,T)$ is \index{T$_1$} {\em \T{1}} if 
$\fall x,y \in X$, with $x \neq y,\ \exst$ open $U$, $V$, $x \in U$, $y \in V$
and $x \not \in V$ and $y \not \in U$. 

It should be easy
to show that this is equivalent to all singletons being closed.\\

\define A topological space $(X,T)$ is \index{T$_{\frac{1}{2}}$}
{\em \T{\frac{1}{2}}} if $\fall x,y \in X$,
with $x \neq y$, either there is an open $U$ about $x$ with $y \not \in U$ or
thre is an open $V$ around $y$ with $x \not \in V$. This is often written as
\index{T$_0$} {\em \T{0}}, also.\\

\define A topological space $(X,T)$ is \index{T$_3$} {\em \T{3}} if 
$\fall x \in X$, $\fall C$
closed in $X$, with $x \not \in C$, there exist open $U$ and $V$ such that $U
\cap V = \emptyset$, $x \in U$, and $C \subseteq V$. Such a space is said to
be \index{regular} {\em regular}.\\

\define A topological space $(X,T)$ is \index{T$_4$} {\em \T{4}} if $fall C,
D$ closed in $X$ with $C \cap D = \emptyset$, there exist open $U,V$ such that
$C \subseteq U$, $D \subseteq V$, and $U \cap V = \emptyset$. Such a space is
said to be \index{normal} {\em normal}.\\

Recall that a Hausdorff space is also known as \T{2}.  It is clear that \T{2}
$\Rightarrow$ \T{1}; and it can be demonstrated that the converse is not true:

Let $X$ be any infinite set and $T$ be $\{\emptyset\} \cup$  $\{X-F \mid F$ is
any finite union of $X \}$. $T$ is a topology on X, called the
\index{topology!finite complement} ``finite complement'' topology. $(X,T)$ is
\T{1}, but not \T{2}. It is \T{1} since all singletons are closed, and it
isn't \T{2} because, other than $\emptyset$, it is impossible to get disjoint
sets.

Now, you may think that from our definition, any set which is \T{3} is also
\T{2}, but this isn't true. The standard definitions of \T{3} and \T{4} assume
that the set is also \T{2}. (Actually, I think you only need to assume \T{1},
and then you can prove \T{2}).

\section{Subspaces}

\define Given a topological space $(X,T)$ with $A \subseteq X$, we can place a
\index{topology!subspace} {\em subspace topology} on $A$, call it $T_A$, defined by $T_A = \{A \cap U \mid U \in T \}$.\\

This is indeed a topology. Firstly, every element of $T_A$ is a subset of $A$,
clearly; so $T_A \subseteq P(A)$. Furthermore,
\begin{enumerate}
\item[$(i)$] $\emptyset \in T_A$, since $\emptyset \in T$ and $A \cap
\emptyset = \emptyset$
\item[$(ii)$] $A \in T_A$, since $X \in T$ and $A \cap X = A$
\item[$(iii)$] $P,Q \in T_A \Rightarrow P \cap Q \in T_A:$
There exist $U,V \in T$ such that $P = A \cap U$ and $Q = A \cap V$, so 
$P \cap Q = (A \cap U) \cap (A \cap V) = (U \cap V) \cap A$.
Since $U \cap V$ is open in $T$, $U \cap V \cap A$ is open in $T_A$.
\item[$(iv)$] $\{Q_\alpha\} \in T_A, \alpha \in I \Rightarrow
\unionover{\alpha}{I}Q_\alpha$: For each $\alpha \in I$, 
$Q_\alpha = A \cap U_\alpha$. where $U_\alpha \in T$. So
$\unionover{\alpha}{I}Q_\alpha = \unionover{\alpha}{I}(A \cap U_\alpha) = A
\cap (\unionover{\alpha}{I}U_\alpha)$. Since $T$ is closed under arbitrary
union, we see that we are intersecting $A$ with an open set in $T$, so $T_A$
is also closed under arbitrary union.
\end{enumerate}

The introduction of subspaces allows for some confusion which might not have
been present previously. If $(X,T)$ is a topological space and $A \subseteq
X$ with subspace topology $T_A$. Given $S \subseteq A$; is $S$ open?  

That question, as asked, is flawed. It is now ambiguous. We could 
be asking ``is $S \in T$?'', or we could be asking ``is $S \in T_A$''? And
there is no way to tell the difference. To resolve this, one must always be
careful to ensure the context is clear. It is sufficient to ask ``is $S$ open
in $X$?'' or ``is $S$ open in $A$?''\\

\example Consider $\reals$ with the usual topology. Let $A = [0,1]$, and 
$S_1 = [0,1]$. Is $S_1$ open in $\reals$? No. But is $S_1$ open in $A$? Yes; in
fact it is, since $S_1 = A \cap \reals$. This demonstrates that the subspace
topology really isn't just the set of open sets which are also subsets of $A$.

Consider $S_2 = (0,1]$. Again, we know that $S_2$ is not open in $\reals$, but
it isn't hard to find a set $U_2 \in \reals$ so that $S_2 = U_2 \cap A$.
In fact, for any $x > 1$, $U_2 = (0,x)$ will work. So $S_2$ is in fact open in
$A$.\\

\begin{theorem}
Given a topological space $(X,T)$ and an open set $A \subseteq X$, and some $S
\subseteq A$, then $S$ is open in $A$ if and only if $S$ is open in $X$.
\end{theorem}

\begin{proof}
$(\Rightarrow)$ $S = A \cap U$, where $U$ is open in $X$. $U \cap A$ is an
intersection of two open sets in, which is in $T$. So $S$ is open in $X$.

$(\Leftarrow)$ Since $S$ is open in $X$, and $S \subseteq A$, $S = A \cap S$,
an open set in $A$.
\end{proof}

In a similar fashion, if there is a $U \in T$ such that $S = X - U$, then we
say $S$ is closed in $X$, and if there is a $V \in A$ such that $S = A - V$,
then we say $S$ is closed in $A$.\\

\example In the previous example, we see that $S_1$ is closed in both $X$ and
$A$, while $S_2$ is not closed in either $X$ or $A$. 

It may not be immediately obvious that 0 is a limit point of $S_2$
using $T_A$. Any open set in $T_A$ about 0 must contain the interval
$[0, \varepsilon)$, for some $\varepsilon > 0$. Since all of these intersect
$S_2$ at some point, 0 is a limit point of $S_2$ in A.\\

Again, even when discussing limit points; note that we had to be explicit
about the context; since in order to consider whether a point is a limit point
or not, we must consider all the open sets around that point. And the
open sets we are considering depend on what topology we are using.\\

Now we have many examples of topologies; since any subset of the real
line can be a topological space.\\

\subsection{The Order Topology}

Let $(X, \leq)$ be a totally ordered set; for example, $\reals$. Let $x < y$
be in $X$. We now formally define \index{interval} the {\em intervals}.
\begin{enumerate}
\item[] $(x,y) = \{z \in X \mid x < z < y \}$
\item[] $(x,y] = \{z \in X \mid x < z \leq y \}$
\item[] $[x,y) = \{z \in X \mid x \leq z < y \}$
\item[] $[x,y] = \{z \in X \mid x \leq z \leq y \}$
\end{enumerate}

The \index{topology!order} {\em order topology} is the topology given by the
basis:
\begin{displaymath}
(x,y) \cup \{X\} \cup (x, M] \cup [m, y)
\end{displaymath}
where $x,y \in \reals$, $x < y$, $M$ is the largest element in $X$, and $m$ is
the least element in $X$. Note that if $X$ does not contain a largest or least
element; then we conveniently ignore the corresponding intervals in the above
expression. We need the $\{X\}$ in the case that $X$ contains only one
point.\\


\exercise The order topology on $\reals$ turns out to be the usual topology.
If $A \subseteq \reals$, then $\leq$ restricted to $A$ makes $A$ a
totally ordered set.  Thus we can impose an order topology on $A$. Prove that
if $\reals$ is given the usual topology, then the order topology on $A$ is the
same as the usual topology on $A$.\\

\section{Topological Interior of a Set}

\define Let $(X,T)$ be a topological space, and $A \subseteq X$. Then the
\index{interior} {\em topological interior} of $A$ is
$A^{\circ} = \{V \subseteq X | V \textrm{ open }, V \subseteq A\}$. 

$\interior{A}$ is the largest subset of $A$ open in $X$, since it is defined to
contain all other such subsets of $A$.\\

\exercise $A^{\circ} = A - \{ \textrm{ limit points of } X - A\}$.\\

\exercise Choose a subset of the $\reals$ and find out how many sets you can
build by repeated operations of closure, complement, and interior. What is the
largest number that can be built?

\section{Metric Spaces}

\define A \index{metric space} {\em metric space} is a pair $(X, d)$ where $X$
is a set and $d$ is a function $d: X \times X \rightarrow \reals$, called the 
\index{distance function} {\em distance function} or the {\em metric}, which
satisfies:
\begin{itemize}
\item[$i.$] $\fall x,y \in X,\ d(x,y) \geq 0$
\item[$ii.$] $\fall x,y \in X,\ d(a,b) = d(b,a)$
\item[$iii.$] $\fall x,y \in X,\ d(a,b) = 0$ if and only if $a = b$
\item[$iv.$] $\fall x,y,z \in X,\ d(a,b) + d(b,c) \geq d(a,c)$
\end{itemize}
Property $iv$ is known as the \index{triangle inequality} {\em triangle 
inequality} (often written \index{\triineq} \triineq).\\

\define Given a metric space $(X,d)$, with some $a \in X$ and
some $\varepsilon \geq 0$, let $\eball(a) = \{x \in X \mid d(a,x) <
\varepsilon\}$. This is called the \index{$\varepsilon$-ball} {\em open
$\varepsilon$-ball} with center at $a$.\\


\claim The collection $\mathcal{B} = \{\eball(a) \mid a \in X,\ \varepsilon > 0\}$ is a
basis for a topology.

\begin{proof}
Recalling Theorem \ref{basisthm}, we must show that $X =
\unionover{\varepsilon}{0}\eball(a)$ and for any $\varepsilon_1 > 0$ and 
$\varepsilon_2 > 0$, $\ball{\varepsilon_1}(a_1) \cap
\ball{\varepsilon_2}(a_2)$ is a union of elements of $\mathcal{B}$.

The first part simple:\\

\noindent
$(\supseteq)$: This is obvious\\
$(\subseteq)$: For any $x \in X$, $x \in \ball{1}(x)$. So the union of all
$\varepsilon$-balls contains every point in $X$.\\

Recalling Lemma \ref{unionlemma}, we need to show
that each $x \in \ball{\varepsilon_1}(a_1) \cap
\ball{\varepsilon_2}(a_2)$  has some $\ball{\varepsilon_3}(a_3)$ such that $x
\in \ball{\varepsilon_3}(a_3) \subseteq (\ball{\varepsilon_1}(a_1) \cap
\ball{\varepsilon_2}(a_2))$. 

Let $\zeta = \min\{\varepsilon_2 - d(x,a_2), \varepsilon_1 - d(x, a_1)\}$. We
claim that $\ball{\zeta}(x) \subseteq (\ball{\varepsilon_1}(a_1) \cap
\ball{\varepsilon_2}(a_2))$.  To demonstrate this, we must show that for any
$z \in \ball{\zeta}(x)$, $d(a_1, z) < \varepsilon_1$ and $d(a_2, z) <
\varepsilon_2$. We have that
\begin{displaymath}
d(z,a_1) \leq d(x, a_1) + d(z,x) < d(x, a_1) + \zeta \leq 
d(x, a_1) + \varepsilon_1 - d(x,a_1) = \varepsilon_1
\end{displaymath}
since $\zeta \leq \varepsilon_1 - d(x,a_1)$. By the same logic $d(z, a_2) <
\varepsilon_2$. So we are done. Notice we used the \triineq $\ $in this proof.
\end{proof}

\define Given a metric space $(X,d)$, the topology defined by $\mathcal{B}$
above is called the \index{topology!metric} {\em metric topology}.\\

\example Consider $(\reals, d)$, where $d(x,y)=|x - y|$. This is known as the
\index{metric!usual} {\em usual metric on $\reals$}. And it is a 
metric:
\begin{itemize}
\item[1.] $|x-y| \geq 0$
\item[2.] $|x-y| = |y-x|$
\item[3.] $|x-x| = |0| = 0$, and $|x| = 0 \iff |x = 0|$ so $|x - y| = 0 \iff x
= y$
\item[4.] \exercise Prove the \triineq. It's easy, but
there are a number of cases to check.
\end{itemize}

The usual metric on $\reals$ leads to the metric topology on $\reals$.
Consider:
\begin{eqnarray*}
\eball(x) & = & \{y \in \reals \mid d(x,y) < \varepsilon\} \\
          & = & \{y \in \reals \mid |x-y| < \varepsilon\} \\
          & = & \{y \in \reals \mid x - \varepsilon < y < x + \varepsilon\} \\
          & = & (x - \varepsilon, x + \varepsilon) \\
          & = & \textrm{ the usual topology on } \reals \\
\end{eqnarray*}
So the metric topology on $\reals$ is the same as the usual topology on
$\reals$; which isn't too surprising.\\

Now consider $\reals^{2} = \reals \times \reals = \{(x,y)|x,y \in \reals\}$.
We can create a few different metrics: 
\begin{eqnarray*}
d_e((x_1,y_1),(x_2,y_2)) & = & \sqrt{(x_2 - x_1)^2 + (y_2 - y_1)^2} \\
d_t((x_1,y_1),(x_2,y_2)) & = & |x_2 - x_1| + |y_2 - y_1| \\
d_s((x_1,y_1),(x_2,y_2)) & = & \max\{|x_2 - x_1|, |y_2 - y_1|\} \\
\end{eqnarray*}

We call $d_e$ the \index{metric!Euclidean} {\em Euclidean metric}, $d_t$ the
\index{metric!Taxi} {\em Taxi metric}, and $d_s$ the \index{metric!$\sup$}
$\sup$ {\em metric}. \\

\exercise Each of these is a metric.\\

\exercise The topologies generated on $\reals^2$ by each of these
metrics is the same.\\

Sketch of proof of above exercise (incomplete; the gist is here, but it needs
to be straightened out): 
In order to avoid confusion, we will refer to $\varepsilon$-balls as
follows:
\begin{itemize}
\item[-] $\eball^t(x)$: an $\varepsilon$-ball around $x$ using the Taxi
metric.
\item[-] $\eball^e(x)$: an $\varepsilon$-ball around $x$ using the Euclidean
metric
\item[-] $\eball^s(x)$: an $\varepsilon$-ball around $x$ using the sup metric.
\end{itemize}

We will demonstrate the following:
\begin{itemize}
\item[1.] Any $\varepsilon$-ball using the taxi metric can be
expressed as a union of $\varepsilon$-balls using the sup metric
\item[2.] Any $\varepsilon$-ball using the sup metric can be expressed as a 
union of $\varepsilon$-balls using the Euclidean metric, and 
\item[3.] Any $\varepsilon$-ball using the Euclidean metric can be expressed
as a union of $\varepsilon$-balls using the taxi metric.
\end{itemize}
Here I am being a bit sloppy, since when I say $\varepsilon$-ball, I don't
mean that the size of the balls will be the same. Putting these together, we
can then express any basis element using any one of the three metrics as a
union of basis elements using either of the other two.

This turns out to be not too hard with the metrics we have. We can show that
$d_s(x,y) \leq d_e(x,y) \leq d_t(x,y) \leq 2d_s(x,y)$ for all $x$ and $y$
in $X$.  This means that, given any $x$,
any $z$ in $\eball^t(x)$ will be contained in $\eball^e(x)$;
and any $z$ in $\eball^e(x)$ will be, in turn, contained in 
$\eball^s(x)$.  Likewise, any $z$ in a $\ball{\frac{\varepsilon}{2}}^s(x)$ 
will be contained in $\eball^t(x)$.  

Consider $\eball^t(x)$ for any $x \in X$. For any $z \in \eball^t(x)$, let
$\delta = \varepsilon - d(x,z)$, so that $\dball^t(x) \subseteq \eball^t(x)$,
but then $\ball{\frac{delta}{2}}^s(z) \subseteq \dball^t(x) \subseteq
\eball^t(x)$. By Lemma \ref{unionlemma}, $\eball^t(x)$ is a union of
elements of the sup topology basis.



Using this, we can show that any $\eball^s(x)$  is a union of
$\varepsilon$-balls in the other two metrics: Given $z \in \eball(x)$, let
$\delta = \varepsilon - d(z,x)$. Then the sup $\dball(x) \subseteq \eball(x)$,
but the 

\exercise To help with the above exercise -- if $d_A$ and $d_B$ are 
two metrics on a space $X$, then what must be true in order for the topologies
to be the same?\\

If, given any epsilon, there is a delta such that the delta-ball with metric B
around a point $x$ is contained entirely within the epsilon-ball with metric
A. Then we know that the metric topology from metric A must be a subset of the
metric topology of metric B. This is because any basis element of A must be in
the topology of metric B; which is shown as follows: given a basis element $U$, an open ball, of A, take any point in that ball. There is an epsilon ball with
ametric A around that point contained entirely within $U$. And there is a
delta ball with metric B contained entirely within that epsilon ball. Thus, by
Lemma \ref{unionlemma}, $U$ is a union of balls from metric B, and is
therefore in the topology imposed by B. 

If $(X,d)$ is a metric space, and $A \subseteq X$, we can restrict $d$ to A to
so that $(A,d|_A)$ is a metric space. The proof is trivial, since each of
the required properties is inherited from $(X,d)$.

Here again we run into the issue of context. If some point $x$ is in $A$, then
to say $\eball(x)$ is ambiguous, since for $(X,d)$, $\eball(x) = \{y \in X \mid
d(x,y) < \varepsilon\}$, while for $(A,d|_A)$, $\eball(x) = \{y \in A \mid
d(x,y) < \varepsilon\}$. (Notice in the condition, I used $d$ instead of
$d|_A$; that's actually fine, since inside $A$, $d = d|_A$.)\\


\define Let $(X,T)$ be a topological space. We say $(X,T)$ is
\index{metrizable} {\em metrizable} if there is a metric $d:X \times X
\rightarrow \reals$ so that $T$ is the metric topology imposed by $d$. 

Note we already know that $d$ is not unique, from a previous exercise. The
usual topology has the Euclidean, taxi, and $\sup$ metrics.\\

\example Let a set $X$ have the discrete topology. Is this space metrizable?

Consider $d: X \times X \rightarrow \reals$ where 
\begin{displaymath}
d(x,y) = \left\{ \begin{array}{ll}

1 & \textrm{if } x \neq y \\
0 & \textrm{if } x = y
\end{array} \right.
\end{displaymath}
It is easy to verify that this is a metric (verification in
notes).

Since $\ball{1}(x) = \{x\}$, we see that the singletons are open; so we do
have the discrete topology. We call this the \index{metric!discrete} {\em
discrete metric}.

\subsection{Facts about Metric Spaces}

\claim Every metric space is \T{2}.

\begin{proof} Let $(X,T)$ be a metric space with metric $d$; and consider $x$
and $y$ in $X$ such that $x \neq y$.  We need to show that there is a $U$ open
about $x$ and a $V$ open about $y$ such that $U \cap V = \emptyset$.

Let $\epsilon  = d(x,y) > 0$. Then let $U = \ball{\frac{\epsilon}{2}}(x)$ and
$V = \ball{\frac{\epsilon}{2}}(y)$.  If $z \in U \cap V$ then $d(x,z) <
\frac{\epsilon}{2}$ and $d(y,z) < \frac{\epsilon}{2}$. But then, by \triineq,
\begin{displaymath}
d(x,y) \leq d(y,z) + d(z,x) < \frac{\epsilon}{2} + \frac{\epsilon}{2}
< \epsilon = d(x,y).
\end{displaymath}
which is obviously a contradiction; so there can be no such $z$, and the
intersection of $U$ and $V$ must be empty. Hence, every metric space is \T{2}.
\end{proof}

\exercise Every metric space is \T{3}.\\

\exercise Every metric space is \T{4}.\\

\define Let $(X,d)$ be a metric space; with $x \in X$ and $A \subseteq X$.
Define $d(x,A) = \inf\{d(x,a) \mid a \in A\}$. This is bounded below by
zero.\\

\exercise If $A$ is closed, then $d(x,A) = 0$ if and only if $x \in A$.\\

Remark: If $x$ has more than one point, the indiscrete topology is not 
\T{2}, and is therefore not metrizable.\\

\begin{lemma}
\label{metricfirstcountablelemma}
Every metric space is first countable.
\end{lemma}

\begin{proof}
Consider $x \in X$; we will show that $x$ has a countable neighborhood base.
We want some set $\{U_i \mid i \in \naturals\}$, where each $U_i$ is open
about $x$ and for any $V$ open about x, there is some $U_i \subseteq V$. So 
let $U_i = B_{\frac{1}{i}}(x)$.

Now, if $V$ is open and $x \in V$, then $x \in \eball(y) \subseteq V$. Let
$\delta = \varepsilon - d(x,y)$, and note $\delta > 0$.\\

We claim that $\dball(x) \subseteq \eball(y)$: Let $z \in \dball(x)$. Then
$d(z,x) < \varepsilon - d(x,y)$. Furthermore, by the \triineq, 
\begin{eqnarray*}
d(z,y) & \leq & d(z,x) + d(x,y) \\
       & \leq & \varepsilon - d(x,y) + d(x,y) \\
d(z,y) & \leq & \varepsilon.
\end{eqnarray*}
So $\dball(x) \subseteq \eball(y)$. By the archimedean property, there is some
$i \in \naturals$ so that $\frac{1}{i} < \delta$; so we have that
\begin{displaymath}
x \in (U_i = \ball{\frac{1}{i}}(x)) \subseteq \dball(x) \subseteq \eball(y)
\subseteq V.
\end{displaymath}
And we see that every metric space is first countable.
\end{proof}

\section{Urysohn's Lemma}
\index{Urysohn's Lemma} 
In notes; need to add

\section{Connectedness}

\define A topological space $(X,T)$ is \index{not connected} {\em not
connected} if there exist nonempty open sets $U$ and $V$ such that $U = X -
V$. I.e. 
\begin{displaymath}
U \cap V = \emptyset \textrm{ and } U \cup V = X \textrm{ and }
U \neq \emptyset \neq V
\end{displaymath}
A space is \index{connected} {\em connected} if it is not not connected.\\

\exercise Show that the indiscrete topology is always connected.\\

\exercise The discrete topology is never connected, unless it is over a
singleton set.\\

\exercise The half-open topology on $\reals$ is not connected. \\

\define An \index{interval} {\em interval} on $\reals$ is any set
$\mathcal{J}$ satisfying $[x < y < z \land x \in \mathcal{J}
\land y \in \mathcal{J}] \Rightarrow y \in \mathcal{J}$.

If we consider all combinations of properties such as ``bounded above''
or not, ``contains least upper bound'' or not, ``bounded below'' or not,
and ``contains greatest lower bound'' or not; we can easily list all types of
intervals on $\reals$.\\

\define If $(X,T)$ is a topological space with $S \subseteq X$, for $S$ to be
connectect means $S$ is connected in the subspace topology on $S$.\\

\claim $S \subseteq \reals$ (with the usual topology) is connected if and only
if S is an interval.\\

\begin{proof}
In notes of 11/17; it's long but I should put it in here. 
\end{proof}

\begin{theorem} The continuous image of a connected space is connected. 

I.e. if $f:(X,T) \rightarrow (Y,S)$ is a continuous function between
topological spaces and if $(X,T)$ is connected and $f$ is onto, then $(Y,S)$
is connected.
\end{theorem}

\begin{proof}
We procede by contrapositive. Assume $f$ is continuous and onto, and that 
$(Y,S)$ is not connected. So we have some $U,V \in S$, both nonempty, such
that $U \cap V = \emptyset$ and $U \cup V = Y$. 

Now, $f^{-1}(U)$ and $f^{-1}(V)$ are connected, since $f$ is continuous. 
Furthermore, $f^{-1}(U) \cup f^{-1}(V) = f^{-1}(U \cup V)$ (this is easily
proven), which in turn is just $f^{-1}(Y) = f^{-1}(X)$ since $f$ is onto.

Also, $f^{-1}(U) \cap f^{-1}(V) = f^{-1}(U \cap V) = f^{-1}(\emptyset) =
\emptyset$. And since $f$ is onto, it and neither of $U$ or $V$ is empty; it
must be that neither of $f^{-1}(U)$ or $f^{-1}(V)$ is onto.

And so we see that if $(Y,S)$ is not connected, then $(X,T)$ is not connected;
equivalently, if $(X,T)$ is connected, then $(Y,S)$ is connected.

\end{proof}

\corollary {\sc to Urysohn's lemma}  If $(X,T)$ is \T{4}, connected, and has
at least two points, then $X$ is uncountable.\\

Before beginning this proof, we should know that if $(X,T)$ is a topological
space with $B \subseteq X$ and $A \subseteq B$, then the subspace topology on
$A$ imposed by $T$ is the same as the subspace topology on $A$ imposed by the
subspace topology on $B$.\\

\exercise Prove that.\\

\begin{proof}
There are $x,y \in X$ such that $x \neq y$, and $\{x\}$ and $\{y\}$ are
closed, since the space is \T{2} and therefore \T{1}, which is equivalent to
singletons being closed. So, by Urysohn's Lemma, there is a continuous
function $f:X \rightarrow [0,1]$ with $f(x) = 0$, $f(y) = 1$ and $f(X)$ is
connected. 

Since $f(X)$ is connected in the subspace topology on $[0,1]$, it
is connected in the usual topology on $\reals$, by the above exercise. Hence,
by the previous proof, it is an interval; and since 0 and 1 are in $f(X)$, it
must be that $[0,1] \subseteq f(X)$, wherein it is evident that $f(X)$ is not
countable, and so $X$ cannot be countable.\\
\end{proof}

\exercise Given topological space $(X,T)$, let ${A_\alpha}$, $\alpha \in I$,
be a family of subsets of $X$. Assume that each $A_\alpha$ is connected and
that there exists an $x \in X$ such that $x \in \intover{\alpha}{I}A_\alpha$.
Prove that $\unionover{\alpha}{I}A_\alpha$ is connected. \\

\define Let $(X,T)$ be a topological space and $x \in X$. The
\index{component} {\em component} of $X$ containing $x$ is
\index{$\mathcal{C}_x$}
$\comp{x} = \bigcup\{A \subseteq X \mid A \textrm{ connected and } x \in
A\}$.\\

We see from the previous exercise that $\comp{x}$ is connected. In fact,
$\comp{x}$ is the largest connected subset of $X$ containing $x$, since if $A
\subseteq X$ and $x \in A$, then $A \subseteq \comp{x}$.\\

\claim A topological space $(X,T)$ is connected if and only if there is only
one component.\\

\begin{proof}
$(\Rightarrow)$: is trivial.

$(\Leftarrow)$: Call the component $\comp{X}$. $\fall x \in X, x \in \comp{X}$,
so $X \subseteq \comp{X}$.
But $\comp{X} \subseteq X$. So $X = \comp{X}$, which is connected.
\end{proof}

\exercise Let $B \subseteq A \subseteq \bar{B}$ in topological space $(X,T)$.
Then if $B$ connected, $A$ is connected.\\

\exercise As corollary to above; components are closed.\\

But are components open?\\

\example let $X = \{0, \frac{1}{2}, \frac{1}{3}, \cdots \}$ in $\reals$. 
Each $\{x\} \subseteq X$ is a component, but $\{0\}$ is not an open set in
$X$.\\

\define \index{disconnected!totally} A topological space is {\em totally disconnected} if the components are the singletons.\\

Discrete $\Rightarrow$ totally disconnected; but the converse is not
necessarily true. Take the case of $\rationals \subseteq \reals$; this set is
totally disconnected but not discrete.  \\

\define In a topological space $(X,T)$ with $S \subseteq X$, we say $S$ is a
\index{separating set} {\em separating set} if $X$ is connected and $X-S$ is
not.\\ 

\example $\{0\}$ is a separating set for $\reals$.\\

\define If $\{x\}$ is a separating set for some topological space $(X,T)$,
then we say $x$ is a \index{separating point} {\em separating point}.\\

\example $0$ is a separating point for $\reals$.\\

\example In $[0,1] \subseteq \reals$ with subspace topology imposed by the
usual topology on $\reals$; every point in $(0,1)$ is a separating point, but
0 and 1 are not separating points.\\

\section{Homeomorphisms}

\define If a $f: (X,T) \rightarrow (Y,S)$ is a function between topological
spaces, we say $f$ is a homeomorphism if it satisfies the following
conditions:
\begin{itemize}
\item[$i.$] $f$ is continuous
\item[$ii.$] $f$ is one to one
\item[$iii.$] $f$ is onto
\item[$iv.$] $f^{-1}$ is continuous
\end{itemize}

Let $X = [0,1)$ and $Y = \{(x,y) \in \reals^{2} \mid x^2 + y^2 = 1 \}$.\\

% Here there should be a FIGURE of a line and a circle.\\

Let $f(x) = (cos(2\pi x), sine(2\pi x)$. Then $f$ is a 1--1 correspondence,
and is continuous. (We will prove this later). But is $f^{-1}$ continuous? For
it to be so, it must be that for every open set $A$ in $X$, $f^{-1})^{-1}(A)$
is open in $Y$. Choose any $[0, \epsilon) \subseteq X$, then
$(f^{-1})^{-1}([0, \epsilon))$ is a half-open interval in Y, and not an open
set. \\

% Again, here would be a good place for a FIGURE to demonstrate. See notes of 11/19. \\


If $f:X\rightarrow Y$ is 1--1 and onto, we get $f:P(X) \rightarrow P(Y)$,
where for $A \subseteq X$, $A \rightarrow f(A)$, where $f(A) \subseteq Y$;
which is also 1--1 and onto.\\

Given two sets $X$ and $Y$, we can impose topology $T$ on $X$ and topology
$S$ on $Y$. Then if $f$ is continuous, then $f^{-1}:S\rightarrow T$; and if
$f^{-1}$ is continous, then $f: T \rightarrow S$. In this case, there is a
1--1 correspondence between open sets.\\

Homeomorphisms are tremendously useful: for two topological spaces $(X,T)$ and
$(Y,S)$, given any topological question, if there is a homeomorphism $f:X
\rightarrow Y$; it will have the same answer for both $X$ and $Y$.\\

\exercise: Assume $f:(X,T) \rightarrow (Y,S)$ is a homeomorphism. Show that 
\begin{displaymath}
(X,T) \textrm{ has property } x \iff (Y,S) \textrm{ has property } x
\end{displaymath}
where x is each of the following: \T{1}, \T{2}, \T{3}, \T{4}, 1$^{st}$
countable, is connected.

Also, show that 
\begin{displaymath}
\mathcal{C} \textrm{ is a component of } X \iff f(\mathcal{C}) \textrm { is a
component of } Y
\end{displaymath}
and
\begin{displaymath}
A \textrm{ is a separating set of } X \iff f(A) \textrm{ is a separating set of } Y
\end{displaymath}

% there's alot more here in the notes... with FIGURE.\\

\section{Compactness} 

\subsection{Covers \& Generating Sets} 
\define In a topological space $(X,T)$, we say that $C \subseteq P(X)$
\index{generate} {\em generates} $T$ if 
\begin{displaymath}
\fall S \subset X\quad S \in T \iff [\fall A \in C,\ S \cap A \textrm{ is open
in } A]
\end{displaymath}
Since $\Rightarrow$ is true by the definition of the subspace topology, when
showing that a set generates a topology, only $\Leftarrow$ requires any
work.\\

\example Consider $\reals$ with the usual topology. Let $C$ be the collection
of all singletons $\{x\}$ in $\reals$.

Note that the topology of $\{x\}$ is well understood; it is $\{\{x\},
\emptyset\}$. In this case the topology is both discrete and indiscrete.

Since for any $\{x\} \in C$ and $S \subseteq X$, $S \cap \{x\}$ is open in
$\{x\}$ regardless of whether $S$ is open or closed; so $C$ does not generate
the usual topology. In fact, it generates the discrete topology.\\

\define Given a topological space $(X,T)$, $C \subseteq P(X)$ is a
\index{cover} {\em cover} of $X$ if $\unionover{A}{C}A=X$.

If every $A \in C$ is open, then $C$ is an open cover; if every $A \in C$ is
closed, then $C$ is a closed cover.\\

\define If $C$ is a cover of $X$ and for all $x \in X$, there exists an open
$U$ about $x$ so that $U \cap A \neq \emptyset$ for only finitely many $A \in
C$, then we say $C$ is a \index{locally finite} {\em locally finite} cover of
$X$.\\

\example $C = \{(i, i+2) \mid i \in \integers\}$ is a locally finite cover of
$\reals$ with the usual topology.\\

\exercise Let $(X,T)$ be a topological space 
\begin{itemize}
\item[$(a)$] {\em (very easy)} Any open over of $X$ generates $T$.
\item[$(b)$] {\em (easy)} Any locally finite closed cover of $X$ generates
$T$.
\end{itemize}
Hint to part $(b)$: $C \subseteq A \textrm{ is closed in A } \iff \exst D
\subseteq X \textrm { closed in X. with } C = D \cap A$.\\

Note: The cover of $\reals$ by singletons is a closed cover, but it is not
locally finite.

\subsection{The Pasting Lemma}

\begin{lemma}[The Pasting Lemma]
Let $f: (X,T)\rightarrow(Y,S)$ be a function between topological spaces, and
let $C \subseteq P(X)$ generate $T$. Then $f$ is continuous if and only if for
all $A \in C$, $f|_A$ is continuous.
\end{lemma}

\begin{proof} 
Exercise.
\end{proof}

\example Here is an application of the Pasting Lemma: Let $f:\reals
\rightarrow \reals$ be defined by 
\begin{displaymath}
f(x)= \left\{\begin{array}{ll} 
 x^2 - 3 & \textrm{ if } x \leq 1 \\
 x - 3 & \textrm{ if } x \geq 1 
 \end{array} \right.
\end{displaymath}
Note that $f$ is well defined, since $1^2 - 3 = 1 - 3$. Furthermore, $f$ is
continuous on $(-\infty, 1]$ and $[1, \infty)$, and $\{(-\infty, 1], [1,
\infty)\}$ is a locally finite closed cover of $\reals$. Hence, $f$ is
continuous on $\reals$.

Note that it was import here that $C$ was a locally finite {\em closed} cover;
since we could have had $f = x + 7$ when $x  > 1$; and we would have still had
a well-defined function, continuous on each of it's parts, whose set of parts
constituted a locally finite cover. But clearly this function would not be
continuous.\\


\subsection{Subcovers \& Refinements} 

Let $C$ and $D$ be two covers of $X$. Then,\\

\define We say $D$ is a \index{subcover} {\em subcover} of $C$ if 
for all $B \in D$, $B \in C$.\\

\define We say $D$ is a \index{refinement} {\em refinement} of $C$ if for all
$B \in D$, there exists an $A \in C$ such that $B \subseteq A$.\\

Notice that every subcover is a refinement, since if $D$ is a subcover of $C$, 
then for any $B \in D$, $B \in C$ and $B \subseteq B$. The converse is not
true.\\

\example Consider the following covers of $\reals$:
\begin{eqnarray*}
C & = & \{(i, i + 7) \mid i \in \integers\} \\
D & = & \{(2i, 2i + 7) \mid i \in \integers\} \\
E & = & \{(x, x + 1) \mid x \in \reals\} 
\end{eqnarray*}
$D$ is a subcover of $C$, but $E$ is not. $E$ is a refinement of $C$.\\

\subsection{Compactness} 

\define A topological space $(X,T)$ is \index{compact} {\em compact} if 
every open cover has a finite subcover.\\

\define A topological space $(X,T)$ is \index{paracompact} {\em paracompact}
if every open cover has a locally finite subcover.\\

\subsubsection{Compactness Lemmas} 

\begin{lemma}
\label{closedcompactlemma}
A closed subset of a compact space is compact.
\end{lemma}
\begin{proof}
Let $C$ be a closed subset of compact space $X$. Let $\{U_\alpha\}$ be an open
cover of $C$. Each $U_\alpha = V_\alpha \cap C$ for some $V_\alpha$ open in
$X$. 

Let $V = X - C$; $V$ is open in $X$. Notice that $(\bigcup V_\alpha) \cup V
= X$, since $(\bigcup V_\alpha) = C$. 

Now, since each $V_\alpha$ is open in $X$, and $V$ is open in $X$, $\{V\} \cup
\{V_\alpha\}$ is an open cover of $x$. And since $X$ is compact, some finite
number of these sets must cover $X$. Say $V_{\alpha_1}, \cdots, V_{\alpha_n}$, and
perhaps $V$.

We claim that $U_{\alpha_1} = V_{\alpha_1} \cap C$, $\cdots$, 
$U_{\alpha_n} = V_{\alpha_n} \cap C$, $\emptyset = V \cap C$,  covers $C$.
{\setlength\arraycolsep{2pt}
\begin{eqnarray*}
(C \cap V_{\alpha_1}) \cup (C \cap V_{\alpha_2}) \cup \cdots \cup (C \cap
V_{\alpha_n}) \cup (C \cap V) & = & \\
C \cap (V_{\alpha_1} \cup \cdots V_{\alpha_n} \cup V) & = & \\
C \cap X & = & C
\end{eqnarray*}
}
So our claim is true; and so $(C \cap V_{\alpha_1}),\ \cdots,\ 
(C \cap V_{\alpha_n}$ is a finite subcover of the original open cover. Hence,
$C$ is compact.
\end{proof}

\example If $T$ in $(X,T)$ is finite, then $(X,T)$ is compact. This may happen
when $X$ is finite, or $T$ is the indiscrete topology.\\

\begin{lemma} 
\label{compacthausdorfflemma}
A compact subset of a Hausdorff space is closed.
\end{lemma}

\begin{proof}
Let $C \subseteq X$, where $C$ is compact and $X$ is \T{2}. We will show that
$X - C$ is open by showing that for any $x \in X - C$, there exists some $U$
open about $x$ with $U \subseteq X - C$.

Given $x \in X - C$, let $y \in C$. So then there are $U_y$ and $V_y$ which
are open, such that $x \in U_y$, $y \in V_y$, and $U_y \cap V_y = \emptyset$.

We claim that there exists a finite set $\{V_{y_1}, V_{y_2}, ..., V_{y_n}\}$
such that $C \subseteq  \bigcup_{i = 1}^{n}V_{y_i}$. This is true because
$\{W_y = V_y \cap C \mid y \in C\}$ is an open cover of $C$, and $C$ is compact.
So there must be a finite subcover. I.e. there are $y_1, \cdots, y_n$ such
that $C = \bigcup_{i = 1}^{n}W_{y_i}$. \\

Now let $U = U_{y_1} \cap \cdots \cap U_{y_n}$. Since $x$ is in each $U_{y_i}$
for $1 \leq i \leq n$, we see that $x \in U$; and furthermore, $U$ is open
since it is a finite intersection of open sets.

We claim that $U \subseteq X - C$. I.e. $U \cap C = \emptyset$. For any
$z \in U \cap C$, there must be some $k$ such that $z \in V_{y_k}$. But also, 
$z \in U_{y_i}$ for $1 \leq i \leq n$, so $z \in U_{y_k}$. But then 
$U_{y_k} \cap V_{y_l} \neq \emptyset$, which is a contradiction \contra.
This proves that a compact subset of a Hausdorff space is
closed.
\end{proof}

\begin{lemma} 
\label{continuouscompactlemma}
The continuous image of a compact space is compact.
\end{lemma}

\begin{proof}
Let $f:(X,T)\rightarrow (Y,S)$ be a function between topological spaces which
is continuous and onto; and let $(X,T)$ be compact.

Let $\{U_\alpha\}$, $\alpha \in I$, be an open cover of $Y$. Note 
that each $f^{-1}(U_\alpha)$ is
open in $X$, since $f$ is continuous.  In fact, $f^{-1}(U_\alpha)$ covers $X$:
\begin{eqnarray*}
\fall x \in X,\ f(x) \in Y & \Rightarrow & \\
f(x) \textrm{ is in some } U_\alpha & \Rightarrow & \\
x \in f^{-1}(U_\alpha)& \Rightarrow & \\
X \subseteq \unionover{\alpha}{I} f^{-1}(U_\alpha) 
\end{eqnarray*}
So $\{f^{-1}(U_\alpha)\}$, $\alpha \in I$, is an open cover of $X$. Since $X$
is compact, there exists an $n \in \naturals$ such that $\{f^{-1}(U_{\alpha_1}),
\cdots, f^{-1}(U_{\alpha_n})\}$ covers $X$.

We claim that $\{U_{\alpha_1}, \cdots, U_{\alpha_n}\}$ covers $Y$. Consider
any $y \in Y$. Since $f$ is onto, there is an $x \in X$ such that $f(x) = y$.
And there is a $U_{\alpha_k}$ such that $x \in f^{-1}(U_{\alpha_k})$, which
means that $y \in U_{\alpha_k}$.  Hence, $Y \subseteq U_{\alpha_1} \cup \cdots
\cup U_{\alpha_n}$, and is therefore compact.
\end{proof}

\begin{lemma} 
\label{compacthausdorffhomeomorphismlemma}
Let $f:(X,T) \rightarrow (Y,S)$ be a continuous 1--1
correspondence between topological spaces, where $X$ is compact and $Y$ is
Hausdorff. Then $f$ is a homeomorphism.
\end{lemma}

\begin{proof}
We need only show that $f^{-1}$ is continuous; i.e. if $U$ is open in $X$,
then $(f^{-1})^{-1}(U) = f(U)$ is open in $Y$.

So, let $U$ be open in $X$. Then $C = X - U$ is closed, and therefore compact
(by Lemma \ref{closedcompactlemma}), and so $f(C)$ is compact (by Lemma
\ref{continuouscompactlemma}), and furthermore, $f(C)$ is closed (by Lemma 
\ref{compacthausdorfflemma}). So $Y - f(C)$ is open.

But since $f$ is a 1--1 correspondence, $Y-f(C) = f(U)$: 
\begin{displaymath}
f(U) \cup f(C) = f(U \cup C) = f(X) = Y
\end{displaymath}
Also, $f(U) \cap f(C) = \emptyset$. If not, then there is some $z \in f(U)
\cap f(C)$. Then there exist $x_1 \in U$ and $x_2 \in C$ where $f(x_1) = z =
f(x_2)$. But since $f$ is 1--1, it must be that $x_1 = x_2$, which means there
is some point $x = x_1 = x_2 \in U \cap C$, a contradition \contra. 

So $f(U) = Y - f(C)$, and $f(U)$ is therefore open. Since for any open set $U$
in $X$, $f(U)$ is open, we know that $f^{-1}$ is continuous, and so $f$ is a
homeomorphism.
\end{proof}

\begin{lemma}
\label{comactt2t3lemma}
If $(X,T)$ is \T{2} and compact, then it is \T{3}.
\end{lemma}

\begin{proof}
Let $x \in X$ and $C \subseteq X$ such that $C$ is closed and $x \nin C$.
We need to find disjoint open sets $U$ and $V$ such that $x \in U$ and $C
\subseteq V$. 

For $y \in C$, let $U_y$, $V_y$ be open and disjoint, where $x \in U_y$ and $y
\in V_y$; these exist since $X$ is \T{2}. Now, since $X$ is compact, there
exists an $n \in \naturals$ such that  $\{V_{y_1}, \cdots, V_{y_n}\}$ covers
$X$, and so $C \subseteq \bigcup_{i = 1}^{n}V_{y_i}$.

Let $U = U_{y_1} \cap \cdots \cap U_{y_n}$, which is open about $x$; and 
$V = V_{y_1} \cup \cdots \cup V_{y_n}$, which is is open and contains $C$. If
there is a point $z \in U \cap V$, then there is a $k$, $1 \leq k \leq n$,
such that $z \in V_{y_k}$, and also $z \in U_{y_i}$ for each $1 \leq i \leq
n$; so $z \in U_{y_k}$. But then $z \in U_{y_k} \cap V_{y_k}$, which is a
contradiction \contra, since these were chosen to be disjoint. Hence there can
be no such $z$, and we see that $U \cap V = \emptyset$. 

Therefore, $U$ and $V$ satisfy the conditions for $X$ to be
\T{3}.
\end{proof}

\exercise If $(X,T)$ is \T{2} and compact, it is \T{4}.\\

\define A topological space $(X,T)$ is \index{Lindel\"of} {\em Lindel\"of} if
every open cover has a countable subcover.\\

\exercise* If $(X,T)$ is \T{2} and Lindel\"of, then it is \T{4}.

\subsection{Sequential \& Limit-point Compactness}

\define A topological space $(X,T)$ is \index{compact!sequentially} {\em
sequentially compact} if every sequence in $X$ has a convergent subsequence.
\\

\example The sequence $a_i = (-1)^i$ has subsequences which may converge to
either $1$ or $-1$. On the other hand, the sequence $b_i = i$ does not have
any convergent subsequences. \\

\define A topological space $(X,T)$ is \index{compact!limit-point} {\em
limit-point compact} if for all $S \subseteq X$ such that $S$ is infinite,
there is an $x \in X$ which is a limit point of $S$. In other words, every
infinite subset of S has a limit point. This is also known as the
\index{Bolzano-Weierstrass} Bolzano-Weierstrass property.

\begin{lemma}
\label{cimplieslclemma}
If a topological space $(X,T)$ is compact, then it is limit-point compact.
\end{lemma}

\begin{proof}
Assume that $X$ is compact, and that $S \subset X$ has no limit points. Since
$S$ has no limit points, it is closed vacuously, and therefore it is also
compact, by Lemma \ref{closedcompactlemma}. 

$S$ is also discrete: since it has no limit points, every $x \in S$ is
isolated. I.e. for any $x \in S$, there is a $U$ open in $X$ such that $U \cap
S - \{x\} = \emptyset$.

Now, if we suppose that $S$ is infite, we arrive at a contradiction! The 
set of singletons in $S$ is an
open cover of $S$ with no subcover of any kind; but we know $S$ is compact,
and so that cover must have a finite subcover. So $S$ must be finite. 

We have proved our claim by contrapositive. If any set with no limit points
must be finite, then any set which is infinite must have at least one limit
point.
\end{proof}

\begin{lemma}
\label{scimplieslclemma}
If a topological space $(X,T)$ is sequentially compact, then it is also
limit-point compact.
\end{lemma}

\begin{proof}
Consider $S \subseteq X$ where $S$ is infinite. Then there is some
$a:\naturals \rightarrow S$ which is 1--1. Since we assume $X$ is sequentially
compact, there must be a convergent subsequence $b$ of $a$; say $b_i
\rightarrow x$. Also, notice that $b$, too, is clearly 1--1.

We claim that $x$ is a limit point of $S$. It suffices to show that every open
$U$ around $x$ contains at least 2 points from $S$; then,  if one of those
points is $x$, the other is not. 

Let $U$ be open about $x$; then there is a $k \in \naturals$ such that $i > k
\Rightarrow a_i \in U$. Now, since $b$ is 1--1, $a_{k+1}$ and $a_{k+2}$ are
two different elements of $S$ which are also in $U$.
\end{proof}


\begin{lemma}
\label{metriclcimpliessclemma}
If a metric space is limit-point compact, then it is sequentially compact.
\end{lemma}

\begin{proof}
Let ($(a_i)$ be a sequence in $X$. Now, either the image of $(a_i)$, $\{a_i
\mid i \in \naturals\}$, call it $I$, is finite, or it is not.

In the first case, say $I =  \{x_1, \cdots,
x_n\}$, and note that for each of the finitely many $x_i$,  $a^{-1}(x_i)
\subseteq \naturals$.  Now, since the union of these $a^{-1}(x_i)$ is
$\naturals$, an infinite set, at least one of the $a^{-1}(x_i)$ must be an
infinite set, since it is impossible for a finite union of finite sets to be
an infinite set. Hence, some $x_i$ appears infinitely times in the sequence.
Let $M = a^{-1}(x_i)$ for that $i$.  The subsequence $a|_M$ converges
to $x_i$, since $a|_M$ is constant. 

In the second case, $I$ is infinite, and by assumption, it has some limit
point; call it $x$. Thus, by lemmas \ref{metricfirstcountablelemma} and
\ref{firstcountablesequencelemma}, there is a sequence in $I$ converging 
to $x$. \\

Before we continue the proof, let us demonstrate a useful fact: \\

\noindent
{\sc Useful Fact:} {\em For any open $U$ about $x$, $U \cap S$ has 
infinitely many elements}. \\

\noindent
{\em Proof of {\sc Useful Fact}.} Since $x$ is a limit point of $S$, any 
open set containing $x$ has some
point in $S$ other than $x$. So, to begin with, choose $\ball{1}(x)$, and
choose some point $s_1 \neq x$. Then let $\delta_1 = d(x,s_1)$, and choose an
element $s_2 \neq x \in \ball{\delta_1}(x)$, and so on ad infinitum. \\

Now, with that in mind, choose an element of $\ball{1}(x) \cap S$ which is not
$x$, say $a_{i_1}$.  Then choose an element of $\ball{\frac{1}{2}}(x)$ which
is not $x$, $a_{i_2}$, where $i_2 > i_1$. We are guaranteed that such an $i_2$
exists because there are only finitely many points between $a_1$ and
$a_{i_1}$, and we have infinitely many points to choose from for $a_{i_2}$.
Continue this process, creating the subsequence $a_{i_j}$ which must converge
to $x$.

We know the sequence converges to $x$ because every open $U$ about $x$
contains some $\eball(x)$, and by the Archimedean property,  there is some 
$\frac{1}{n} < \varepsilon$ so that $U$ contains $ \ball{\frac{1}{n}}(x)$
and for every $j > n$, $a_{i_j} \in \ball{\frac{1}{n}}$. 

We have therefore found a convergent subsequence, and our proof is complete.
\end{proof}

\begin{lemma}
If a metric space $(X,d)$ is sequentially compact, then it is compact.
\end{lemma}

\begin{proof}
We procede by contraction. Assume that $X$ is sequentially compact and not
compact. Let $D = \{U_\alpha\}$ be an open cover of $X$ with no finite
subcover. For any $x \in X$, let $E_x = \{ \varepsilon > 0 \mid 
\exst U_\alpha \in D \textrm { with } \eball(x) \subseteq U_\alpha \}
\subseteq R$. Finally, let $\varepsilon(x) = \frac{1}{2}$ if $E_x$ is
unbounded and
$\frac{\textrm{lub }E_x}{2}$ if $E_x$ is bounded. We choose $\varepsilon(x)$ in this
way so that $\ball{\varepsilon(x)}(x)$ is always contained in some member of
$D$, and either it is  equal to $\frac{1}{2}$ or $\ball{3\varepsilon(x)}(x)$
is contained in no element of $D$.

Now choose $x_1 \in X$. Around $x_1$, there is 
$\ball{\varepsilon(x_1)}(x_1) \subseteq U_{\alpha_1}$. Note that 
$U_{\alpha_1} \neq X$, since $X$ has no finite subcover of $D$.
Then choose $x_2 \in X - \ball{\varepsilon(x_1)}(x_1)$. For $x_2$,
$\ball{\varepsilon(x_2)}(x_2) \subseteq U_{\alpha_2}$, and again 
$U_{\alpha_2} \cap U_{\alpha_1} \neq X$. 
Continue in this manner (with $x_3 \in X - (\ball{\varepsilon(x_1)}(x_1) \cup
\ball{\varepsilon(x_2)}(x_2))$, and so on). Since no finite number of
$U_\alpha$ will cover $X$, this process will
construct a sequence in $X$. By hypothesis, there is a subsequence,
call it $x_{i}$, which converges to some limit $L$.  Let us examine
$\ball{\varepsilon(L)}(L) \subseteq U_\alpha \in D$.\\


% Here would be a good place for a FIGURE\\

There is some $k \in \naturals$ such that for all $i > k$, $x_i \in
\ball{\frac{\varepsilon(L)}{90}}(L))$. We didn't need to pick a number as
large as 90, but it works fine for our purposes. 

Now, let $p = x_{k+1}$ and $q = x_{k+2}$. Both are in
$\ball{\frac{\varepsilon(L)}{90}}(L)$, and $q \nin \ball{\varepsilon(p)}(P)$
because that is how we constructed our sequence. By the triangle inequality, 
\begin{displaymath}
d(p,q) \leq d(p,L) + d(L,q) < \frac{\varepsilon(L)}{90} +
\frac{\varepsilon(L)}{90} = \frac{\varepsilon(L)}{45}.
\end{displaymath}
And so it must be that $\varepsilon(p) < \frac{\varepsilon(L)}{45}$, and $3\varepsilon(p) <
\frac{\varepsilon(L)}{15}$. \\

Now, for all $z \in \ball{3\varepsilon(p)}(p)$, $d(z, p) < 3\varepsilon(p) <
\frac{\varepsilon(L)}{15}$, and so, by the triangle inequality, 
\begin{displaymath}
d(z,L) \leq (\frac{\varepsilon(L)}{15} + \frac{\varepsilon(L)}{90} =
\frac{4\varepsilon(L)}{90}) < \varepsilon(L).
\end{displaymath}

But then $\ball{3\varepsilon(p)}(p) \subseteq \ball{\varepsilon(L)}(L) \subseteq
\textrm{ some } U_\alpha \in D$. \contra. This is a contradiction, since
either $\varepsilon(p) = \frac{1}{2}$ or
$\ball{3\varepsilon(p)}(p)$ is contained in no $U_\alpha$. We know that is
is contained in some $U_\alpha$, and also
$\varepsilon(p)$ cannot be $\frac{1}{2}$: it is strictly less than
$\varepsilon(L)$ (by a large margin, the way we chose it), which in turn is
less than or equal to $\frac{1}{2}$. So we see that if a metric space is
sequentially compact, then it is also compact.
\end{proof}

The above lemmas lead directly to the following fact:
\begin{lemma}
\label{scicislc}
In a metric space $(X,d)$, the following are equivalent:
\begin{itemize}
\item[$i.$] $X$ is compact
\item[$i.$] $X$ is limit-point compact
\item[$ii.$] $X$ is sequentially compact
\end{itemize}
\end{lemma}

\exercise Given $a_i:\naturals \rightarrow \reals$ a sequence; prove that
there is a monotonic subsequence.\\

\subsection{Compactness in $\reals$}

\claim $[0,1]$ is compact.\\

\begin{proof}
Let $S \subseteq [0,1]$ be an infinite set. Let $a_0 = 0$ and $b_0 = 1$, so
that $S \subseteq [a_0, b_0]$. Now let $m_0 = \frac{a_0 + b_0}{2} =
\frac{1}{2}$. Now, $[a_0, m_0] \cup [m_0, b_0]$ contains all of $S$, so at
least one of these intervals must contain infinitely many points of $S$.
Choose the one that does (if they both do, then either will work) and 
call it $[a_1, b_1]$.  (So, for example, in the case that  $[\frac{1}{2}, 1]$ 
contained infinitely many points in S, we might choose $a_1 = \frac{1}{2}$ and 
$b_1 = 1$). 

Now, let $m_1 = \frac{a_1 + b_1}{2}$, and repeat this process. We see that for
all $i \in \naturals$, $[a_i, b_i]$ contains infinitely many points of $S$;
$a_i \geq a_{i-1}$ and $b_i \leq b_{i-1}$, and $b_i - a_i = \frac{1}{2^i}$. 

So we have created two bounded monotone sequences, $a_i \rightarrow A$ and
$b_i \rightarrow B$. We claim that $A = B$: Consider $\varepsilon > 0$.
There exists a $j \in \naturals$ so that $\frac{1}{2^j} <
\frac{\varepsilon}{16}$, and there exist $k_1$ and $k_2$ such that when $i >
k_1$, $d(a_i, A) < \frac{\varepsilon}{16}$ and when $i > k_2$, $d(b_i, B) <
\frac{\varepsilon}{16}$. Now let $N = \max\{j, k_1, k_2\}$, and for any $i >
N$, we have that 
\begin{displaymath}
d(A,B) \leq d(A,a_i) + d(a_i, b_i) + d(b_i, B) \leq \frac{3\varepsilon}{16} <
\varepsilon
\end{displaymath}
So for any positive $\varepsilon$, $d(A,B) < \varepsilon$, so it must be 0;
and so, $A = B$.

Now let $L = A = B$. We claim $L$ is a limit point of $S$. For any $U$ open
about $L$, there is a $\delta$ such that $(L - \delta, L + \delta) 
\subseteq U$. We have the following: 
\begin{eqnarray*}
a_i \rightarrow L & \Rightarrow & \exst P \in \naturals \st i > P \Rightarrow
a_i \in (L - \delta, L + \delta) \\
b_i \rightarrow L & \Rightarrow & \exst Q \in \naturals \st i > Q \Rightarrow
b_i \in (L - \delta, L + \delta)
\end{eqnarray*}
So, choosing $i > \max\{P,Q\}$, we have that $[a_i,b_i] \subseteq (L - \delta,
L + \delta)$, and $[a_i, b_i]$ contains infinitely many points of $S$, so we
are done. We have proved that $[0,1]$ is compact in the usual topology on
$\reals$.
\end{proof}

The above fact is useful, since it allows us to demonstrate that any 
closed interval is compact. This follows from the fact that for any 
$a,b \in \reals$ with $a < b$, $[a,b]$ is the continuous image $[0,1]$, using
the function
\begin{displaymath}
f(x) = xb + (1-x)a.
\end{displaymath}
Since the continuous image of a compact space is compact (Lemma 
\ref{continuouscompactlemma}), we are done.

Furthermore, any closed subset of a closed interval $[a,b]$ is compact,
since any closed subset of a compact space is compact (Lemma 
\ref{closedcompactlemma}).  This, in fact, describes all
compact sets in $\reals$, and is summed up in the following lemma: 

\begin{theorem}[(Heine-Borel)]
Any $S \subseteq \reals$ is compact if and only if it is closed and bounded.
\end{theorem}

\begin{proof}
$(\Leftarrow)$: If $S$ is closed and bounded, let $m$ be a lower bound and $M$
an upper bound. Then $S$ is a closed subset of $[m,M]$,
a closed interval; and so $S$ is compact.

$(\Rightarrow)$: We can cover any compact $S \subseteq \reals$ 
with 1-balls: $C = \{\ball{1}(x) \mid x \in S \}$ is an open cover. 
There exists a finite subcover $\{\ball{1}(x_1), \ball{1}(x_2), \cdots,
\ball{1}(x_n)\}$ for $S$. Let $M$ be the largest of $x_1$, $\cdots$, $x_n$, and
$m$ be the smallest. Then $S \subseteq [m-1, M+1]$; $m-1$ is a lower
bound, and $M+1$ is an upper bound. So $S$ is bounded. Since $S$ is 
compact subset of a Hausdorff space, it is also closed, by Lemma
\ref{compacthausdorfflemma}.

Note that this does not work for all metric spaces!
\end{proof}

\section{Product Spaces}
\subsection{The Product Topology}

\subsubsection{Finding a Good Topology}

Let $(X,T)$ and $(Y,S)$ be topological spaces. Consider the \index{product}
{\em product} $(X \times Y) = \{(x,y) \mid x \in X, y \in Y\}$. What is a
reasonable topology to place on this set?\\
\\
Parallel questions one might ask are:
\begin{itemize}
\item[-] What functions with domain $(X \times Y)$ would you like to be
continuous?
\item[-] What functions with range $(X \times Y)$ would you like to be
continuous?
\end{itemize}
\vspace{12pt}
\define The \index{projection} \index{$\pi_x$} {\em projections} from $(X \times
Y)$ to its factors are:
\begin{displaymath}
\pi_x: (X \times Y) \rightarrow X \textrm{ defined by } \pi_x(a,b) = a
\end{displaymath}
and
\begin{displaymath}
\pi_y: (X \times Y) \rightarrow Y \textrm{ defined by } \pi_y(a,b) = b
\end{displaymath}\\
\\
Now consider $X \times Y$ as a range. Take $f: Z \rightarrow X \times Y$. For
each $z \in Z$, $f(z) \in X \times Y$. I.e $f(z) = (a,b)$ for $a \in X$ and $b
\in Y$.  It is as if there are two functions ``inside'' the function $f$: 
\begin{displaymath}
f_x:Z \rightarrow X \textrm{ where } f_x(z) = a
\end{displaymath}
and
\begin{displaymath}
f_y: Z \rightarrow Y \textrm{ where } f_y(z) = b. 
\end{displaymath}
Actually,  $f_x = \pi_x \circ f$  and $f_y = \pi_y \circ f$. 

We can turn this idea around: Let $h: Z \rightarrow X$ and
$j: Z \rightarrow Y$ be given. We can now create $k: Z \rightarrow (X \times
Y)$ by $k(z) = (h(z), j(z))$ and note that $\pi_x \circ k = h$ and $\pi_y \circ
k = j$. \\

% FIGURE\\

We claim that $k$ is the only function such that $\pi_x \circ k = h $ and
$\pi_y \circ k = j$. Assume some other function $g$ satisfies $\pi_x \circ g =
h$ and $\pi_y \circ g = j$. What is $g(z)$? Say $g(z) = (c,d)$. Then 
\begin{displaymath}
c = \pi_x(c,d) = \pi_x \circ g(z) = h(z) 
\end{displaymath}
and 
\begin{displaymath}
d = \pi_y(c,d) = \pi_y \circ g(z) = j(z)
\end{displaymath}
so $h(z) = c$ and $j(z) = d$, whereby $g(z) = (h(z), j(z)) = k(z)$. \\

Given $f: Z \rightarrow (X \times Y)$, the functions $\pi_x \circ f$ and
$\pi_y \circ f$ are called the \index{coordinate functions} {\em coordinate
functions} of $f$. \\

We now introduce a ``{\sc Universally Accepted Desire}'' of the topology on
$(X \times Y)$; which is to say, this is the condition which will make the
topology on the product space useful to topologists. In our topology, we
want any function with range $(X \times Y)$ to be continuous if and only
if it's coordinate functions are continuous.

We can think of this biconditional as a test. If a topology satisfies this
condition, it may be useful to us. There is such a topology, and we call it
the \index{topology!product} {\em product topology}.\\http://lolocaust.sonnyd.net/army-setup.php

Consider the function $f:(X \times Y) \rightarrow (X \times Y)$ defined by
$f(x,y) = (x,y)$. Certainly this function must be continuous in any topology,
since for all $A \subseteq (X \times Y)$ $\inv{f}(A) = A$; so if $A$ is open,
$\inv{f}(A)$ is of course also open. If $f$ is continuous, then we must make
sure that $\pi_x \circ f$ and $\pi_y \circ f$ are also continuous. \\

% FIGURE \\

Since $f$ is the identity, $\pi_x \circ f = \pi_x$ and $\pi_y \circ f =
\pi_y$. Thus, in order to make the coordinate functions continuous, each
of $\pi_x$ and $\pi_y$ must be continuous. That is, each
of $\inv{\pi_x}(U)$ for $U \in T$ and $\inv{\pi_y}(V)$ for $V \in S$
must be open in $X \times Y$.

There exists a smallest topology on $X \times Y$ making this condition true.
This is the topology given by the subbasis 
\begin{displaymath}
\{\inv{\pi_x}(U) \mid U \in T\} \cup \{\inv{\pi_y}(V) \mid V \in S\}.
\end{displaymath} 
Call this topology $P$.\\


It turns out that $P$ is the product topology. Consider the topological space 
$(X \times Y, P)$. 
As mentioned, $P$ is the smallest topology on $(X \times Y)$ making
$\pi_x$ and $\pi_y$ continuous. 

Now consider the topological space  $(X \times Y, \textrm{ "product top." })$, 
which also makes
$\pi_x$ and $\pi_y$ continuous. Since $P$ is the smallest such topology, 
we know that every element of $P$ is in the product topology. 
Now consider the function between these spaces,
\begin{displaymath}
f:(X \times Y, P) \rightarrow (X \times Y, \textrm{ "product top." }) \st
f(x,y) = (x,y).
\end{displaymath}

Again, $\pi_x \circ f = \pi_x$ and $\pi_y \circ f = \pi_y$, and by assumption, 
these functions are continuous in the product topology, which means that $f$
has to be continuous. So, for 
any $U$ in the product topology, $\inv{f}(U) = U$ must be in $P$. This says
that every element of the product topology is in $P$. So the two are one in 
the same.

This means that any topology imposed on $(X \times Y)$ which makes every $f: Z
\rightarrow X \times Y$ continuous exactly when $\pi_x$ and $\pi_y$ are
continuous must be $P$, the product topology.\\

\exercise If $P$ is the topology used on $(X \times Y)$, then any $f:(X \times
Y)$ is continuous if and only if both $\pi_x \circ f$ and $\pi_y \circ f$ are
continuous.\\

\subsubsection{Basis for the Product Topology}

Given topological spaces $(X,T)$ and $(Y,S)$, a basis for the product
topology on $(X \times Y)$ would be the set of all finite intersections
of $E = \{\inv{\pi_x}(U) \mid U \in T \} \cup \{ \inv{\pi_y}(V) \mid V \in
S\}$. This leads to the question, what do elements in this basis for the
product topology look like?\\

Consider any two elements $O_1$ and $O_2$ of $E$; we want to understand 
\mbox{$O_1 \cap O_2$}. There are 3 possibilities: 
\begin{itemize}
\item[Case I] $O_1 = \inv{\pi_x}(U_1)$ for some $U_1 \in T$ and 
$O_2 = \inv{\pi_x}(U_2)$ for some $U_2 \in T$. Then $O_1 \cap
O_2 = \inv{\pi_x}(U_1 \cap U_2) \in E$, since $U_1 \cap U_2 \in T$.

\item[Case II] $O_1 = \inv{\pi_y}(V_1)$ for some $V_1 \in S$ and 
$O_2 = \inv{\pi_y}(V_2)$ for some $V_2 \in S$. Again, we have nothing new
here.

\item[Case III] $O_1 = \inv{\pi_x}(U)$ for some $U \in T$ and $O_2 =
\inv{\pi_y}(V)$ for some $V \in S$. In this case, 
\begin{eqnarray*}
\inv{\pi_x}(U) & = & \{(x,y) \in (X \times Y) \mid \pi_x(x,y) \in U\} \\
               & = & \{(x,y) \in (X \times Y) \mid x \in U\} \\
               & = & (U \times Y).
\end{eqnarray*}
In similar fashion, $\inv{\pi_y}(V) = (X \times V)$. And we see that
\begin{displaymath}
\inv{\pi_x}(U) \cap \inv{\pi_y}(V) = \{(x,y) \in (X \times Y) \mid x \in U
\land y \in V\} = U \times V.
\end{displaymath}.\\
\end{itemize}

% A FIGURE would be good here.\\

In all cases, we have the product of an open set in $X$ and an open set in
$Y$. Let $B = \{ U \times V \mid U \in T \land V \in S \}$, and consider 
$(U_1 \times V_1) \cap (U_2 \times V_2)$ for some $(U_1 \times V_1)$ and $(U_2
\times V_2)$ in $B$.

We claim that 
\begin{displaymath}
(U_1 \cap U_2) \times (V_1 \cap V_2) = (U_1 \times V_1) \cap (U_2 \times V_2).
\end{displaymath}
Indeed,
\begin{eqnarray*}
(U_1 \times V_1) \cap (U_2 \times V_2) 
 & = & \{(x,y) \mid x \in U_1 \land
                    x \in U_2 \land y \in V_1 \land y \in V_2 \} \\
 & = & \{(x,y) \mid x \in (U_1 \cap U_2) \land y \in (V_1 \cap V_2)\} \\
 & = & (U_1 \cap U_2) \times (V_1 \cap V_2)
\end{eqnarray*}.

So $B$ is closed under intersection, which means $B$ contains all finite
intersections of elements of $E$; and $B$ is the basis of the product
topology.\\

To sum up, any open set in the product topology is a union of sets of 
the form $U \times V$ where $U$ is open in $X$ and $V$ is open in $Y$.\\

\example $\reals^2 = \reals \times \reals$, and it turns out that the
product topology on $\reals^2$ is the same as the topology for the sup metric
(which we have show is the same as the topologies for the Taxi and Euclidean
metrics).\\

\exercise The basis $B$ may also be written as
$\{U \times V \mid (U,V) \in T \times S\}$. Given a topological space $(X,T)$
with basis $B_1$ and another topological space $(Y,S)$ with basis $B_2$; then
a valid basis for the product topology is $\{U \times V \mid (U,V) \in B_1
\times B_2\}$.\\

\subsubsection{Product Topology on Arbitrary Dimensions}

We may extend the concept of product topologies to arbitrary dimensions:\\

Let $(X_\alpha, T_\alpha)$ be an indexed family of topological spaces with
indexing set $I$. Then 
\begin{displaymath}
\productover{\alpha}{I}X_\alpha = \{f: I \rightarrow
\unionover{\alpha}{I}X_\alpha \mid \fall\alpha \in I,\ f(\alpha) \in
X_\alpha\}.
\end{displaymath} 

Of course, we have coordinate functions here, too.
For any $\beta \in I$, there is a function $\pi_\beta:
\productover{\alpha}{I}\rightarrow X_\beta$ given by $\pi_\beta(f) =
f(\beta)$ ($\pi_\beta$ is the projection from $\productover{\alpha}{I}X_\alpha$
to $X_\beta$. 

\index{topology!product, arbitrary}
In $\productover{\alpha}{I}X_\alpha$ consider 
$E = \unionover{\alpha}{I}\{\inv{\pi_\alpha}(U) \mid U \in T_\alpha\}$; the
product topology on $\productover{\alpha}{I}X_\alpha$ is the one given by the
subbasis $E$. \\

\exercise The product topology as defined above on
$\productover{\alpha}{I}X_\alpha$ has the property that if $(Z,Q)$ is a
topological space, then
\begin{displaymath}
g: Z \rightarrow \productover{\alpha}{I}X_\alpha \textrm{ is continuous } \iff
\fall \beta \in I,\ \pi_\beta \circ g : Z \rightarrow X_\beta \textrm{ is 
continuous }
\end{displaymath}

\subsubsection{Basis for Product Topology on Arbitrary Dimensions}

Now, what about the basis for the product topology? Again, let $(X_\alpha,
T_\alpha)$ be a family of topological spaces indexed over $I$, and consider 
$\productover{\alpha}{I}X_\alpha$ with the product topology, with subbasis $E$
as indicated above. In order to get a basis, we will need all finite
intersections. Recall,
\begin{displaymath}
\productover{\alpha}{I}X_\alpha = \{f:I\rightarrow
\unionover{\alpha}{I}X_\alpha \mid \fall \alpha \in I,\ f(\alpha) \in
X_\alpha\}.
\end{displaymath}
And
\begin{eqnarray*}
\inv{\pi_\beta}(U) & = & \{f:I\rightarrow \unionover{alpha}{I}X_\alpha \mid
(\fall \alpha \in I,\ f(\alpha) \in X_\alpha) \land \pi_\beta(f) \in U \subseteq
X_\beta\} \\
& = & \{f:I\rightarrow \unionover{\alpha}{I}X_\alpha \mid (\fall \alpha \neq
\beta,\ f(\alpha) \in X_\alpha) \land f(\beta) \in U\} \\
& = &  U \times \productover{(\alpha \neq \beta)}{I}X_\alpha.
\end{eqnarray*}\\

I.e., $\inv{\pi_\beta}(U)$ is the product of the following subsets of
$X_\alpha$, $\alpha \in I$:
\begin{displaymath}
\left\{\begin{array}{lll}
       \textrm{ when } \alpha & = & \beta \textrm{ use subset } U\\
       \textrm{ when } \alpha & \neq & \beta \textrm{ use } X_\alpha
       \end{array}
\right\}
\end{displaymath}\\


Any finite intersection of such elements, 
\begin{displaymath}
\inv{\pi_{\beta_1}}(U_1) \cap \cdots \cap \inv{\pi_{\beta_n}}(U_n)
\end{displaymath}
would be a product of the following subsets of $X_\alpha$...
\begin{displaymath}
\left\{\begin{array}{lll}
       \textrm{ if } \alpha   =  \beta_1 &\textrm{ use }& U_1 \\
       \textrm{ if } \alpha  = \beta_2 & \textrm{ use }& U_2 \\
         &\vdots & \\
       \textrm{ if } \alpha  = \beta_n & \textrm{ use }& U_n \\
       \textrm{ otherwise} & \textrm{ use } & X_\alpha
       \end{array}
\right\}
\end{displaymath}
If two $U_i$ come from the same $X_\alpha$, then you must use the intersection
which is another element of that $X_\alpha$.

These form the basis for the product topology: all products of open sets in
$X_\alpha$ where {\em all but finitely} many are the $X_\alpha$ themselves.\\

\subsubsection{Example: The Product Topology on the Set of Real Sequences}
\index{topology!on collection of sequences}

\begin{displaymath}
\reals^\infty = \reals^1 \times \reals^1 \times \cdots =
\productover{i}{\naturals}X_i
\end{displaymath}
where each $X_i = \reals$.
\begin{eqnarray*}
\productover{i}{\naturals}X_i 
& = & \{f:\naturals \rightarrow \unionover{X_i} 
        \mid \fall i \in \naturals,\ f(i) \in X_i\} \\
& = & \{f: \naturals \rightarrow \reals 
        \mid \fall i \in \naturals,\ f(i) \in \reals\} \\
& = & \{f:\naturals \rightarrow \reals\} \\
& = & \textrm{ all real sequences.}
\end{eqnarray*}
So we now have a product topology, $\reals^\infty$,
on the collection of real sequences.\\

Let $(a_i)$ be a real sequence. That is, $(a_i) \in \reals^\infty$. A basic open
set in $\reals^\infty$ about $a_i$ is ... (really need a picture for this one)
... a set which for finitely many $i \in \naturals$ restricts the value of the
the $a_i$ to an open set in $\reals$ about $a_i$, but for the rest of $i$, is
completely unrestricted.

% FIGURE

Essentially, the open set can be seen as a finite number of ``gates'' through
which a sequence must fit in order to be an element of that set. The important
point to grasp is that we can force a sequence to be  ``very close'' to
$(a_i)$ in only  finite many places.\\

By way of contrast, there is another topology, called the \index{topology!box}
{\em box topology} (box top for short), on $\productover{\alpha}{I}X_\alpha$
where basic open sets look like $\productover{\alpha}{I}U_\alpha$ where for all
$\alpha$, $U_\alpha$ is open in $X_\alpha$. We will discuss this more later.\\

\subsubsection{Facts about Product Spaces}

Recall, $f: Z \rightarrow X \times Y$ is continuous if and only if $\pi_x
\circ f$ and $\pi_y \circ f$ are continuous.\\

% FIGURE 

Consider $p \in \productover{\alpha}{I}X_\alpha$, so 
$p:I\rightarrow \unionover{\alpha}{I}X_\alpha$ such that for all $\alpha$, 
$p(\alpha) \in X_\alpha$. Now consider $\beta \in I$; let 
\begin{displaymath}
C_\beta(p) = \{f \in \productover{\alpha}{I}X_\alpha \mid \fall \alpha \neq
\beta,\ f(\alpha) = p(\alpha)\}.
\end{displaymath}
Recall, a function is really just point, an ``ordered arbitrarily long
tuple''. So $C_\beta(p)$ is actually the set of point which contains $p$, and
contains every other point for which all values except in the $\beta$
direction are the same as those in $p$. Another way to say it is that a slice
is always a ``copy'' of $X_\beta$ in the product space, which contains $p$.
So if $X_\beta$ is a line, the slice will be a line; if it is a plane, the
slice will be a plane, etc.

In two dimensions, say $X \times Y$, it helps to think of a slice as $\{x_0\}
\times Y$ for some $x_0 \in X$.



\exercise $\pi_\beta|_{C_{\beta}:C_\beta(p) \rightarrow X_\beta(p)}$ is
a homeomorphism.\\

We can call $C_\beta(p)$ \index{slice} the {\em slice} parallel to $X_\beta$
through $p$.\\

\exercise If $X_\alpha$, $\alpha \in I$, is a family of topological spaces,
then 
\begin{displaymath}
\productover{\alpha}{I}X_\alpha \textrm{ is connected } \iff \fall \alpha \in
I,\ X_\alpha \textrm{ is connected }
\end{displaymath}
$(\Rightarrow)$ is trivial, since the continuous image of a connected set is
connected. $(\Leftarrow)$ is more interesting.\\

\begin{lemma}
If $X_\alpha$, $\alpha \in I$, is a family of topological spaces, then
\begin{displaymath}
\productover{\alpha}{I}X_\alpha \textrm{ is compact } \iff \fall \alpha \in
I,\ X_\alpha \textrm{ is compact }.
\end{displaymath}
\end{lemma}

Again, $(\Rightarrow)$ is trivial since the continuous image of a compact set
is compact. Before we discuss $(\Leftarrow)$, some terminology:\\

\define A function $f: (X,T) \rightarrow (Y,S)$ is called
\index{function!open} {\em open} if for all $U \in T,\ f(U) \in S$. 

Note that this does not mean that $\inv{f}$ is continuous, because $\inv{f}$
may not even be a function. From this, we can say that a homeomorphism is a
function which is 1--1, onto, and open.\\

\exercise $\pi_\beta: \productover{\alpha}{I}X_\alpha \rightarrow X_\beta$ is
open.\\

Consider the problem in two dimensions: we can project the open cover of
$X \times Y$ to an open cover of $X$ and an open cover of $Y$, but now we
want to do the converse. From an open cover of $X$ and an open cover of $Y$,
we want to construct an open cover of $X \times Y$.

% FIGURE FIGURE FIGURE

A first attempt might be to take an open cover, find a finite subset that 
covers $X$, and a finite subset that covers $Y$, and take the union of those
two subsets. This will likely fail though, since there really is no guarantee
that this will cover all of $X \times Y$. (A
visual aid would be very helpful here. Look in the notes.)\\

\begin{proof}
(For finite dimensions): Given an open cover $\mathcal{C}$ of $X \times Y$, consider any $x_0$ in $X$. 
Then $\{x_0\} \times Y$ is a slice parallel to the $Y$ axis. Now take the set
of all $A \in \mathcal{C}$ which contain $(x_0,y)$ for any $y \in Y$. The set
of projections of these $A$ onto $Y$ must cover $Y$ since each $(x_0,y)$ must be
contained in some $A$.  Thus, there is a finite subset which also covers $Y$.

Now we have a finite set $A_1, \cdots, A_n$, each of which contain $(x_0,y)$
for some $Y$. Projecting these onto $X$, we have finitely many sets $U_1,
\cdots, U_n$, all of which are open in $X$ and which contain $x_0$.
Intersecting these, we obtain another open neighborhood around $x_0$; call it
$N_{x_0}$. Notice that $N_{x_0} \times Y$ is covered by $A_1, \cdots, A_n$. We
call $N_{x_0} \times Y$ an open \index{tube} {\em tube} about $x_0$.

When we do the above for every $x$ in $X$, the set of neighborhoods creates an
open covering of $X$,
which must have a finite subcovering: $N_1, \cdots, N_n$. For each $N_i$, the
tube 
$N_i \times Y$ is covered by finitely many elements of $C$. And the set
of neighborhoods $N_i$ covers $X$, so we have finitely many elements of $C$
which cover $X \times Y$.

The proof for any finite number of dimension follows by induction. For
arbitrary dimensions, additionaly work is required: see the Tychonoff Theorem.
\end{proof}

\subsubsection{Topologies on Sequences}

Recall, $\reals^\infty = \reals \times \reals \times \reals \times \cdots$ is
the set of all real sequences.

Consider the sequence $\vec{0} = (0, 0, 0, \cdots) \in
\reals^\infty$, and 
\begin{eqnarray*}
x_1 & = & (1, \frac{1}{2}, \frac{1}{3}, \cdots)\\
x_2 & = & (\frac{1}{2}, \frac{1}{4}, \frac{1}{6}, \cdots)\\
x_3 & = & (\frac{1}{3}, \frac{1}{6}, \frac{1}{9}, \cdots)\\
    & \vdots &
\end{eqnarray*}

Does $\vec{x_i} \rightarrow \vec{0}$ in the product topology?  \\

Yes. For any $\varepsilon > 0$, there is an $N_0$ such that beyond all $N_0$, 
all $x_{i_1}$ are within $\varepsilon$ of 0. Remember, we can only construct
finitely many ``gates'', so given those gates, choose the largest $i \in N$
such that there is a gate at $i$.  This is called
\index{convergence!pointwise} {\em pointwise convergence}.

What about in the box topology? It turns out that almost nothing converges in
the box topology, because you can make any restriction you want an any point
you want. 

Between the product topology and the box topology is the
\index{topology!uniform} {\em uniform topology}, which allows you to restrict
any and all $i \in N$, but they must all be restricted by the same value
$\varepsilon$.





\printindex
\end{document}
