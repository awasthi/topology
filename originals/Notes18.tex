\documentclass[11pt,reqno]{amsart}
\usepackage{amsmath}
\usepackage{amsthm}
\usepackage{amssymb}
\usepackage{amsfonts}
\usepackage{latexsym}
\usepackage{verbatim}
\usepackage{graphicx}

\setlength \oddsidemargin{.5in}
\setlength \evensidemargin{.5in}
\setlength \textwidth{5.5in}
\newcommand{\bmat}[1]{\left[\begin{array}{#1}}
\newcommand{\bbmat}[0]{\end{array}\right]}
\newcommand{\vect}[1]{\mathbf{#1}}
\newcommand{\pmat}[0]{\begin{pmatrix}}
\newcommand{\ppmat}[0]{\end{pmatrix}}
\newcommand{\Ker}[0]{\text{Ker }}
\newcommand{\Img}[0]{\text{Im }}
\newcommand{\Null}[0]{\text{Null }}
\newcommand{\Rank}[0]{\text{rank }}

\newcommand{\R}[0]{\mathbb{R}}
\newcommand{\Q}[0]{\mathbb{Q}}
\newcommand{\N}[0]{\mathbb{N}}
\newcommand{\C}[0]{\mathbb{C}}
\newcommand{\Z}[0]{\mathbb{Z}}
\newcommand{\B}[0]{\mathbb{B}}
\newcommand{\M}[0]{\mathbb{M}}
\newcommand{\U}[0]{\mathbb{U}}
\newcommand{\F}[0]{\mathbb{F}}
\newcommand{\W}[0]{\mathbb{W}}


\newcommand{\inner}[1]{\left\langle #1 \right\rangle}
\newcommand{\norm}[1]{\left|\left| #1 \right|\right|}

\newcommand{\frm}[1]{\mathcal{F}_{#1}}
\newcommand{\prv}[1]{\mathcal{P}_{#1}}

\newcommand{\myImage}[2]{\begin{figure}[h] \centering \includegraphics[width={#1}]{#2} \end{figure}}
\newcommand{\myImageC}[3]{\begin{figure}[h] \centering \includegraphics[width={#1}]{#2} \caption{#3} \end{figure}}

\newcommand{\seq}[2]{\{{#1}_{#2}\}}

\newcommand{\AND}[0]{\wedge}
\newcommand{\OR}[0]{\vee}
\newcommand{\NOT}[0]{\neg}
\newcommand{\IMPLIES}[0]{\rightarrow}
\newcommand{\DEDUCE}[0]{\vdash}
\newcommand{\myEquiv}[0]{\equiv_\llcorner}
\newcommand{\Bslash}[0]{B \! \! \! \! \! \diagup}

\newcommand{\myBox}[3]{\framebox[#1][c]{\parbox{#2}{#3}}}

%%% Theorem Styles
\newtheorem{Proposition}{Proposition}
\newtheorem{Corollary}{Corollary}
\newtheorem{Theorem}{Theorem}
\newtheorem*{Thm}{Theorem}
\newtheorem{Lemma}{Lemma}
\theoremstyle{definition}
\newtheorem*{Definition}{Definition}
\newtheorem{Example}{Example}
\newtheorem*{Remark}{Remark}
\newtheorem*{Question}{Question}


\allowdisplaybreaks

\begin{document}
    \begin{center}
    \LARGE{Topology Notes} \\
    \vspace{.1in}
    \normalsize{10 March 2010}\\
    \vspace{.1in}
    \normalsize{Anna Bessesen} \\
    \vspace{.2in}
    \Large{Ch. 8: Hausdorff} \\
    \end{center}
    \Definition
    Let $(X, F _X)$ be a topological space. We say that X is \emph{Hausdorff} if $\forall$ $p, q \in X$ such that $p \neq q$, $\exists$ disjoint sets $U, V \in F_X$ such that $p \in U$, $q \in V$. \\
    \Example
    Metric spaces \textbf{are} Hausdorff.\\
    \Example
    (Non-Example)
    Any space containing at least two points under the indiscrete topology is \textbf{not} Hausdorff.\\
    
    \textbf{Flapan Says...} Hausdorff is important because any \textquotedblleft reasonable\textquotedblright space/any space that we want to be in will be Hausdorff.\textquotedblright \\
    
    \textbf{Note:} Continuous functions \emph{may not} preserve Hausdorff-ness.\\
    \Example
    $f$$\colon$($\R$, discrete) $\to$ ($\R$, indiscrete) by the identity.\\
    
    \indent
    This is continuous because the domain has the discrete topology, but the domain is Hausdorff while the range clearly is not.\\
    
    \textbf{Small Fact about Hausdorff:}
    Suppose $f \colon (X, F_X)$ $\to$ $(Y, F_Y)$ is a homeomorphism and $(X, F_X)$ is Hausdorff. Then $(Y, F_Y)$ is also Hausdorff.\\
    
    \begin{proof} Let $p \neq q \in Y$. Because f is a bijection, $f^{-1}(p)$ and $f^{-1}(q)$ are distinct points in X. \\
    \indent
    Since $X$ is Hausdorff, $\exists$ $U, V \in F_X$ such that $U \cap V = \emptyset$, $f^{-1}(p) \in U$, $f^{-1}(q) \in V$. \\
    \indent
    Consider $f(U)$, $f(V)$. Since $f$ is a homeomorphism and, thus, open, $f(U)$ and $f(V)$ are open. Then, because $f$ is a bijection, we have $p \in$ $f(U)$, $q \in$ $f(V)$, and $f(U) \cap f(V)$ = $\emptyset$
    \end{proof}
    
    We are now interested in how Hausdorff-ness and compactness are related.\\
    
    \indent
     \textbf{Flapan Says...} \textquotedblleft Hausdorff-ness and compactness are like two people who love each other, because when two people love each other, they can do so much more together than they can alone.\textquotedblright \\
     \\
     \indent
     \textbf{Theorem:} Let $(X, F_X)$ be Hausdorff. Let A $\subseteq$ X be compact. Then A is closed.\\
     
     \begin{proof} (Compare the following to the proof that all compact subsets of metric spaces are closed. That is, we are going to show that we don't need a metric, just Hausdorff-ness, for closed subsets to be compact.)\\
     \indent
     Rather than show $A$ is closed, we will show $X-A$ is open.\\
     \indent
     Let $p \in X-A$. We want to show $\exists$ $U \in F_X$ such that $p \in U \subseteq X-A$.\\
     \indent
     Let $a \in A$. Because X is Hausdorff, $\exists$ $U_a, V_a \in F_X$ such that $p \in U_a$, $a \in V_a$, $U_a \cap V_a = \emptyset$.\\
     \indent
     Consider $\{ V_a | a \in A\}$. This is an open cover of A because $V_a$ open and $a \in V_a$ $\forall a \in A$. So, because A is compact, $\exists$ finite subcover $\{ V_{a_1}, V_{a_2}, ..., V_{a_n}\}$\\
     \indent
     Let $U = \bigcap_{i=1}^{n} U_{a_i}$. $U \in F_X$ since it is a finite intersection of elements of $F_X$ and $p \in U$ since $p \in U_{a_i}$ $\forall$ $i=1, 2, ...,$ $n$.\\
     \indent
     \textbf{Claim:} $U \subseteq X-A$\\
     \emph{Proof.} $\forall$ $i=1, 2, ...$ $n$, $U_{a_i} \cap V_{a_i} = \emptyset$\\
     \indent
     $A \subseteq \bigcup_{i=1}^{n} V_{a_i}$\\
        \indent
     $U \cap A$ $\subseteq$ $U \cap \bigcup_{i=1}^{n} V_{a_i}$\\
     \indent
     However, $U \cap \bigcup_{i=1}^{n} V_{a_i} = \emptyset$ because if $\exists$ $x \in U \cap \bigcup_{i=1}^{n} V_{a_i}$, then $\exists$ $i_o$ such that $x \in V_{a_{i_o}} \cap U_{a_{i_o}}$. But, by definition, $V_{a_{i_o}} \cap U_{a_{i_o}} = \emptyset$, so no such $x$ exists. Thus,\\
     \indent
     $U  \cap A$ $\subseteq$ $U \cap \bigcup_{i=1}^{n} V_{a_i} = \emptyset$, meaning $U \cap A = \emptyset$. So, because $U \subseteq X$ but $U \cap A = \emptyset$, $U \subseteq X-A$\\
     \indent
     Thus, we have found $U \in F_X$ such that $p \in U$, $U \subseteq X-A$. Thus, $X-A$ is open.\\
     \indent
     Thus, \textbf{A is closed}.
     \end{proof}
     
     \textbf{Corollary:} In any Hausdorff space, (finite sets of) points are closed sets.\\
     \begin{proof} Let $p \in (X, F_X)$ and let $(X, F_X)$ be Hausdorff. Let $\{ U_i|i \in I\}$ be an open cover of $\{ p\}$. Then $\exists$ $i_o \in I$ such that $p \in U_{i_o}$. Thus $\{ U_{i_o} \}$ is a finite subcover. Hence, $\{ p \}$ is compact.\\
     \indent
     Thus, by Theorem above, $\{ p\}$ is closed.
     \end{proof}
     
     $\textbf{Important Lemma}$ about Homeomorphisms (using Hausdorff): Let\\ $f \colon (X, F_X)$ $\to$ $(Y, F_Y)$ be continuous. Let $X$ be compact, and $Y$ be Hausdorff. Then $f$ is a homeomorphism if and only if it is a bijection.\\
     \begin{proof}
     ($\Rightarrow$) Since $f$ is a homeomorphism, $f$ is a bijection.\\
     ($\Leftarrow$) Suppose $f$ is a continuous bijection.\\
     \indent
     WTS $f$ is open. Since $f$ is a bijection, this is equivalent to showing $f$ is closed.\\
     \indent
     Let $A \subseteq X$ be closed. Since $X$ is compact, by Analogous Theorem 2, $A$ is compact. By Analogous Theorem 1, $f(A)$ is compact. By above Theorem, $f(A)$ is closed. Thus, $f$ is closed, and, thus, open. Thus, $f$ is an $open,$ $continuous$ $bijection$.\\
     \indent
     Thus, $f$ is a homeomorphism.\\
     \indent
     Thus, $f$ is a homeomorphism if and only if it is a bijection.
     \end{proof}
     
\end{document}