\documentclass{article}

\usepackage{amssymb, amsmath, amsthm, graphicx}

\newtheorem{theorem}{Theorem}
\newtheorem{corollary}[theorem]{Corollary}
\newtheorem{definition}{Definition}

\parskip=.2in
\parindent=0in

\title{Topology Notes}
\author{Alec Boyd}
\date{January 25th 2010}


%----------
\begin{document}
\maketitle

$\mathbf{Theorem:}$ Let $(M_{1},d_{1})$ and $(M_{2},d_{2})$ be metric space and $f:M_{1}\rightarrow M_{2}$.  Then $f$ is continuous iff for every open set $U\subseteq M_{2}$, $f^{-1}(U)$ is open in $M_{1}$.

$\mathbf{Proof:}$  $(\Rightarrow )$ ...done in previous class.

$(\Leftarrow)$ Suppose that for all open $U\subseteq M_{2}$, $f^{-1}$ is open in $M_{1}$.  Let $p \in M_{1}$.  We want to show that $f$ is continuous at $p$.

Let $\epsilon > 0$.  $B_{\epsilon}(f(p))$ is open, so $f^{-1}(B_{\epsilon}(f(p)))$ is open.  $p\in f^{-1}(B_{\epsilon}(f(p)))$, so $\exists$ $\delta>0$ such that $B_{\delta}(p)\subseteq f^{-1}(B_{\epsilon}(f(p)))$.

 $f(B_{\delta}(p))\subseteq f(f^{-1}(B_{\epsilon}(f(p))))=B_{\epsilon}(f(p))$, so $f$ is continuous at $p$, and thus everywhere.  $\blacksquare$
 
 We've proven the...
 
 $\mathbf{small fact:}$  If $f : M_{1} \rightarrow M_{2}$, then $f$ is continuous if for all $p\in M_{1}$, and for all $\epsilon>0$ $f^{-1}(B_{\epsilon}(f(p)))$ is open.
 
 
 \large{Chapter 2} \normalsize
 
 For this portion of this lecture, we side-step the concept of openness, and define a topology as follows.
 
 $\mathbf{Definition:}$  Let $X$ be a set and $F$ be some collection of subsets of $X$ such that
 	
		$1)$ $X,\O \in F$.
		
		$2)$ If $U,V \in F$ then $U\cap V \in F$ .
		
		$3)$ If for all $i\in I$, $U_{i} \in F$, then $\cup U_{i} \in F$.
		
	Then we say that $(X,F)$ is a topological space with open sets the elements of $F$.  We also 		say $F$ is the  $\emph{topology}$ on $X$.
	

$\mathbf{Ex:}$  Let $(M,d)$ be a metric space and $F$ be the set of open sets in $M$.  Then $(M,F)$ is a topological space.

$\mathbf{Sub-Example:}$  Let $M$ be a set and $d$ be the discrete metric, then we say $(M,F)$ is the $\emph{discrete topology}$.

$\mathbf{Ex:}$  Let $X$ be a set with at least 2 points.  Let $F=\{X,\O\}$.  Then we say $(X,F)$ is the $\emph{indiscrete}$, or $\emph{concrete}$ topology.

$\mathbf{Definition:}$  If $F_{1}$ and $F_{2}$ are topologies on $X$ and $F_{1}\subseteq F_{2}$ then we say that $F_{1}$ is $\emph{weaker}$ than $F_{2}$, or $F_{2}$ is $\emph{stronger}$ than $F_{1}$.

Note:  $weaker = fewer = coarser$ and $stronger = more =finer$.

Note: The $\emph{discrete}$ topology is the strongest topology on $M$.  The $\emph{indiscrete}$ topology is the weakest topology on $X$.

$\mathbf{Ex:}$  Let $X=R$ and $U\in F$ iff $U$ is the union of sets of the form $[a,b)$ such that $a,b\in \mathbb{R}$.  This is called the half-open interval topology.

Is this weaker, stronger, or neither compared to the usual topology?

Is $(a,b) \in F$?  If so, then $F$ is stronger.

Let $a,b \in \mathbb{R}$, and $a<b$.  Then, $(a,b)=\bigcup_{n\in \mathbb{N}}[a+\frac{1}{n},b)$.

$\mathbf{Ex:}$ Let $X=\mathbb{R}^{2}$ have the "dictionary order".  This means that $(a,b)<(c,d)$ if either $a<c$ or a=c and $b<d$.

$U\in F$ iff $U$ is a union of "open intervals", i.e. $\{(x,y)|(a,b)<(x,y)<(c,d)\}$.

.

.

.

$\mathbf{Q:}$ Is a vertical line open?  

.

.

.

.

Answer: Yes.

$\mathbf{Q:}$  Is a horizontal line open?

.

.

.

.

Answer: No.

$\mathbf{Q:}$  Is this finer or coarser than the usual topology on $\mathbb{R}^{2}$?

.

.

.

.

Any point in a ball in the usual topology can be found in a ball of the dictionary topology, which is contained in the usual ball.  Open balls are open in the dictionary topology, so the dictionary topology is finer than the usual topology.

Note: In $\mathbb{R}$, $\bigcup_{n\in \mathbb{z}}(n,n+1)$ is open.

$\mathbf{Q:}$  Are the topologies on a given set always linearly ordered?

 \end{document}