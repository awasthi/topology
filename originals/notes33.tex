\documentclass[11pt,reqno]{article}
\usepackage{ulem}
\usepackage{amsmath}
\usepackage{amsthm}
\usepackage{amssymb}
\usepackage{amsfonts}
\usepackage{latexsym}
\usepackage{graphicx}
\usepackage[T1]{fontenc}
\usepackage[latin1]{inputenc}
\usepackage[extra]{tipa}

%%  Renewed

\renewcommand{\Re}[1]{\operatorname{Re} #1 }
\renewcommand{\Im}[1]{\operatorname{Im} #1}

%% New Commands
\newcommand{\dD}{\partial \mathbb{D}}
\newcommand{\Z}{\mathbb{Z}}
\newcommand{\D}{\mathbb{D}}
\newcommand{\R}{\mathbb{R}}
\newcommand{\Q}{\mathbb{Q}}
\newcommand{\C}{\mathbb{C}}
\newcommand{\A}{\mathcal{A}}
\newcommand{\K}{\mathbb{K}}
\renewcommand{\P}{\mathbb{P}}
\newcommand{\N}{\mathbb{N}}
\newcommand{\cl}{\operatorname{cl}}
\newcommand{\ran}{\operatorname{ran}}
\newcommand{\norm}[1]{\| #1 \|}
\newcommand{\inner}[1]{\left< #1 \right>}
\newcommand{\blf}{ {[\,\cdot\, , \,\cdot\,]} }
\newcommand{\h}{\mathcal{H}}
\newcommand{\M}{\mathcal{M}}
\newcommand{\E}{\mathcal{E}}
\newcommand{\V}{\mathcal{V}}
\newcommand{\W}{\mathcal{W}}
\newcommand{\T}{\mathbb{T}}
\newcommand{\dom}{\mathcal{D}}
\newcommand{\pc}{\perp_C}
\newcommand{\vecspan}{\operatorname{span}}
\newcommand{\interior}{\operatorname{int}}
\newcommand{\lcm}{\operatorname{lcm}}
\newcommand{\tr}{\operatorname{tr}}
%%  Matrices
\newcommand{\minimatrix}[4]{\begin{pmatrix} #1 & #2 \\ #3 & #4 \end{pmatrix}  }
\newcommand{\megamatrix}[9]{\begin{pmatrix} #1 & #2 & #3 \\ #4 & #5 & #6 \\ #7 & #8 & #9\end{pmatrix}  }

\renewcommand{\hat}{\widehat}
\renewcommand{\labelenumi}{(\roman{enumi})}

\newcommand{\twovector}[2]{\begin{pmatrix} #1\\#2 \end{pmatrix} }
\newcommand{\threevector}[3]{\begin{pmatrix} #1\\#2\\#3 \end{pmatrix} }

\renewcommand{\vec}[1]{{\bf #1}}

\newcommand{\varnot}{\sim\!\!}
\newcommand{\due}[1]{\vspace{-0.2in}\begin{center}\textsc{due at the beginning of class \underline{#1}} \end{center}\medskip }

%%%
%%% Theorem Styles
%%%
\newtheorem{Proposition}{Proposition}
\newtheorem{claim}{Claim}
\newtheorem{Corollary}{Corollary}
\newtheorem{Theorem}{Theorem}
\newtheorem*{Thm}{Theorem}
\newtheorem{Postulate}{Postulate}
\newtheorem{Lemma}{Lemma}
\theoremstyle{definition}
\newtheorem*{Definition}{Definition}
\newtheorem*{Example}{Example}
\newtheorem*{Remark}{Remark}
\newtheorem{Exercise}{Exercise}
\newtheorem*{Question}{Question}
\newtheorem{Problem}{Problem}
\newtheorem*{soln}{Solution}


\allowdisplaybreaks
%\newcounter{ex}[section]



%\numberwithin{section}{chapter}
%\numberwithin{Theorem}{chapter}
%\numberwithin{equation}{chapter}
%\numberwithin{Example}{chapter}

%%
%% MAIN DOCUMENT
%%
\begin{document}
\author{James Buerger}
\date{}

\title{Topology Notes for  April 26th, 2010}
\maketitle

\begin{Theorem}  Let $p:\widetilde{X}\rightarrow X$ be a covering map.  Then
\begin{enumerate}
\item  $p$ is an open map;
\item $X$ is a quotient space, and $p$ is a quotient map.
\end{enumerate}
\end{Theorem}

\begin{proof} First things first:
\begin{enumerate}
\item Let $U$ be open in $\widetilde{X}$.  We want to show that $p(U)$ is open.  Let $x \in p(U).$  There exists an evenly covered open set $V$ containing $X$.  Let $y \in U$ such that $p(y)=x.$  Then $p^{-1}(V)=\displaystyle\bigcup_{\alpha \in A}V_{\alpha}$ such that $V_{\alpha}$ are disjoint open sets and $p \mid V_{\alpha}$ is a homeomorphism for all $\alpha \in A$.  Let $\alpha_0 \in A$ such that $y \in V_{\alpha_0}$, where such an $\alpha_0$ exists because $x \in V$.
\\\\\\\\\\\\\\\\\\\
Because $U$ and $V_{\alpha_0}$ are both open, $U\cap V_{\alpha_0}$ is open in $\widetilde{X}$. Recall $p\mid V_{\alpha_0}:V_{\alpha_0}\rightarrow V$ is a homeomorphism, so 
\begin{align*}
p(U\cap V_{\alpha_0})&=p\left(V_{\alpha_0}\cap p(U)\right)\\
&=V\cap p(U)
\end{align*}
where $V\cap p(U)$ is open in $V$ and $U$ is open in $X$. Therefore, $V\cap p(U)$ is open in $X$. Because our $x$ is in both $V$ and $p(U)$, $x\in V\cap p(U)\subseteq p(U)$. That is, $x$ is an element of an open set contained in $p(U)$. Hence, $p(U)$ is open and $p$ is an open map.
\item To show $X$ has the quotient topology with regard to $p$, we want to show that $F_X = \{U\subseteq X \mid p^{-1}(U)\in F_X\}$.

$(\subseteq)$ Let $V\in F_X$. Because $p$ is continuous, $p^{-1}(V)\in F_{\widetilde{X}}$. Thus, $V\in \{U\subseteq X \mid p^{-1}(U)\in F_X\}$.

$(\supseteq)$ Let $U\subseteq X$ such that $p^{-1}(U)\in F_{\widetilde{X}}$. Because $p$ is open, $p(p^{-1}(U)$ is open in $X$. Because $p$ is onto $p(p^{-1}(U))=U$. Thus, $U\in F_X$.

$F_X = \{U\subseteq X \mid p^{-1}(U)\in F_X\}$, so $p$ is a quotient map and $X$ is a quotient space. \end{enumerate} \end{proof}
\begin{Definition}
Let $p:\widetilde{X}\rightarrow X$ be a covering map and $f:Y\rightarrow X$ be continuous. We define a \textit{lift} of $f$ to be any continuous function $\widetilde{f}:Y\rightarrow X$ such that $p\circ \widetilde{f}=f$.
\end{Definition}

\begin{Example}
Let $\widetilde{X}=\R$, $X=S^1$ and $p(x)=\left(\cos(2\pi x),\sin(2\pi x)\right)$. Let $f:I\rightarrow S^1$ by $f(x)=\left(\cos(\pi x),\sin(\pi x)\right)$. Define $\widetilde{f}:I\rightarrow \R$ by $f(x)=\frac{x}{2}$.
\\\\\\\\\\\\\\\\\\
\end{Example}
\begin{Lemma}
\textbf{Uniqueness of Lifts.} Let $p:\widetilde{X}\rightarrow X$ be a covering map and $f:Y\rightarrow X$ be a covering and $f:Y\rightarrow X$ be continuous and $Y$ be connected. Let $\widetilde{f_0}$ and $\widetilde{f_1}$ be lifts of $f$. Suppose there exists $y_0\in Y$ such that $\widetilde{f_0}(y_0)=\widetilde{f_1}(y_0)$ then $\widetilde{f_0}=\widetilde{f_1}$.
\end{Lemma}
\begin{proof}
Let $Y' =\{y\in Y\mid \widetilde{f_0}(y)=\widetilde{f_1}(y)\}$. $Y\ne\emptyset$. We want to show that $Y' = Y$, we will accomplish by showing $Y'$ is clopen in $Y$. 
\begin{itemize}
\item\textit{Open.} Let $y\in Y'$. There exists an evenly covered open set $V$ containing $f(y)$. $p^{-1}(V)=\displaystyle\bigcup_{\alpha\in A} V_\alpha$ such that $V_{\alpha}$'s are disjoint and open; also, $p\mid V_\alpha:V\alpha\rightarrow V$ is a homeomorphism. 

Let $q= \widetilde{f}_0(y)=\widetilde{f_1}(y)$. There exists and $\alpha_0\in A$ such that $q\in V_{\alpha_0}$. $\widetilde{f}_0^{-1}(V_{\alpha_0})$ and $\widetilde{f}_1^{-1}(V_{\alpha_0})$ are open in $Y$, and $y\in \widetilde{f}_0^{-1}(V_{\alpha_0})\cap\widetilde{f}_1^{-1}(V_{\alpha_0})$.
We claim that $\widetilde{f}_0^{-1}(V_{\alpha_0})\cap\widetilde{f}_1^{-1}(V_{\alpha_0})\subseteq Y'.$

Let $z\in \widetilde{f}_0^{-1}(V_{\alpha_0})\cap\widetilde{f}_1^{-1}(V_{\alpha_0}),$ implying $\widetilde{f_0}(z)\in V_{\alpha_0}$ and $\widetilde{f_1}(z)\in V_{\alpha_0}$. Because $\widetilde{f_0}(z)$ and $\widetilde{f_1}(z)$ are lifts of $f$, $p\circ\widetilde{f_0}(z)=f(z)$ and $p\circ\widetilde{f_1}(z)=f(z)$. Since $p$ is 1-1 (because $p$ is a homeomorphism) and $\widetilde{f_0}(z)\in V_{\alpha_0}$ and $\widetilde{f_1}(z)\in V_{\alpha_0}$, $\widetilde{f_0}(z)=\widetilde{f_1}(z)$. Thus, $z\in Y'$.

Hence, $\widetilde{f}_0^{-1}(V_{\alpha_0})\cap\widetilde{f}_1^{-1}(V_{\alpha_0})$ is an open subset of $Y'$ containing $y$, so $Y'$ is open.
\item\textit{Closed.} To show $Y'$ is closed, we will show $Y-Y'$ is open. Let $y\in Y-Y'$. here exists an evenly covered open set $V$ containing $f(y)$. $p^{-1}(V)=\displaystyle\bigcup_{\alpha\in A} V_\alpha$ such that $V_{\alpha}$'s are disjoint and open; also, $p\mid V_\alpha:V\alpha\rightarrow V$ is a homeomorphism. 

Note that $\widetilde{f}_0(y)\ne\widetilde{f_1}(y)$. There exists and $\alpha_1,\alpha_2\in A$ such that $\widetilde{f}_0(y)\in V_{\alpha_1}$ and $\widetilde{f}_1(y)\in V_{\alpha_2}$. The set $\widetilde{f}_0^{-1}(V_{\alpha_1})\cap\widetilde{f}_1^{-1}(V_{\alpha_2})$ is open and contains $y$. We claim that $\widetilde{f}_0^{-1}(V_{\alpha_1})\cap\widetilde{f}_1^{-1}(V_{\alpha_2})\subseteq Y-Y'$.

Let $z\in \widetilde{f}_0^{-1}(V_{\alpha_1})\cap\widetilde{f}_1^{-1}(V_{\alpha_2}),$ implying $\widetilde{f_0}(z)\in V_{\alpha_1}$ and $\widetilde{f_1}(z)\in V_{\alpha_2}$. We now want to show that $\widetilde{f_0}(z)\ne\widetilde{f_1}(z)$. Recall that $p\mid V_{\alpha_1}$ is $1-1$ and $p\circ\widetilde{f}_0(y)=f(y)=p\circ\widetilde{f}_1(y)$: because $\widetilde{f}_0(y)\ne\widetilde{f}_1(y)$ where $\widetilde{f}_0(y)\in V_{\alpha_1}$ and $\widetilde{f}_1(y)\in V_{\alpha_2}$, implying that $\alpha_1\ne \alpha_2$. Otherwise, $\widetilde{f}_0(y)$ and $\widetilde{f}_1(y)$ would be two points in $V_{\alpha_1}$ that both map to $f(y)$. Therefore, $V_{\alpha_1}\cap V_{\alpha_2}=\emptyset$.

Therefore $\widetilde{f_0}(z)\ne\widetilde{f_1}(z)$, implying that $z\in Y-Y'$. This implies $y$ is contained in the open set $\widetilde{f}_0^{-1}(V_{\alpha_1})\cap\widetilde{f}_1^{-1}(V_{\alpha_2})\subseteq Y-Y'$, making $Y-Y'$ open. 
\end{itemize}
Therefore $Y'$ is clopen in $Y$. Because $Y'$ is non-empty and $Y$ is connected, $Y'$ must be all of $Y$. By the definition of $Y'$, $\widetilde{f_0}=\widetilde{f_1}$.
\end{proof}
\end{document}







